\documentclass[10pt,a4paper]{article}
\usepackage[latin1]{inputenc}
\usepackage[spanish]{babel}
\usepackage{amsmath}
\usepackage{amsfonts}
\usepackage{amssymb}
\usepackage{graphicx}
\usepackage[left=4cm,right=3cm,top=4cm,bottom=3cm]{geometry}
\author{Macroeconom�a II}
\title{Dinero e Inflaci�n}
\date{25 de Mayo del 2019}
\begin{document}
\maketitle
\begin{enumerate}
\item Supondremos que las familias decidir�n cuanto dinero demandar para el consumo presente y futuro. Luego el problema de las familias es:
\begin{equation}
\max_{c_{t},m_{t},b_{2}} U(c,m): \left[ logC_{1}+log\dfrac{M_{1}}{P_{1}}+\beta\left( logC_{2}+log\dfrac{M_{2}}{P_{2}} \right)  \right] 
\end{equation}
\begin{align*}
s.a&\\
c_{1}&=Y_{1}-\dfrac{b_{2}}{1+\bar{r_{1}}}-m_{1}\\
c_{2}&=Y_{2}+b_{2}-m_{2}+\dfrac{m_{1}}{1+\pi_{2}}
\end{align*}
donde $ m_{t}=\dfrac{M_{t}}{P_{t}} $, con $ M_{t} $ es la oferta nominal de dinero propuesta por el Banco Central y $ (1+\bar{r_{1}})=(1+r_{1})-\pi_{2} $.
\begin{itemize}
\item[a.]Suponga un mundo de dos periodos $ (T=2) $. Escriba el problema que enfrenta las familias, y encuentre las condiciones de primer orden, para la restricci�n de primer periodo utilice $ \lambda_{1} $ y para la segunda restricci�n utilice $ \lambda_{2} $. Adem�s, considere que $ b_{1}=b_{3}=0 $ y $ m_{0}=0 $
\item[b.]Demuestre que la demanda de saldos reales para el primer y segundo periodo son $ m_{1}=c_{1}\left[ \dfrac{1+r_{1}}{r_{1}}\right]  $ y $ m_{2}=c_{2} $ respectivamente. Interprete el como reacciona la demanda de saldos en el primer periodo ante un cambio en la tasa de inter�s nominal.
\item[c.]Utilizando $ r=\bar{r}+\pi_{2} $, reescriba la demanda de dinero del primer periodo en funci�n de la tasa de inter�s real y la tasa de inflaci�n esperada. C�mo reacciona la demanda ante variaciones en la inflaci�n esperada?. Ayuda: suponga que $\beta\left( 1+\bar{r}\right) =1$, y que el Banco Central fija $ M_{1} $ y decide $ M_{2} $. �C�mo se ajusta la neutralidad del dinero en el an�lisis anterior?
\end{itemize}

\item Usando el siguiente modelo:
\begin{center}
$ \max U: \dfrac{C_{1}^{1-\sigma}}{1-\sigma}+\beta\dfrac{C_{2}^{1-\sigma}}{1-\sigma}+\lambda_{1}\left( Y_{1}+\dfrac{M_{1}}{P_{1}}-\dfrac{B_{2}}{P_{2}}\left( \dfrac{1}{1+i}\right)\dfrac{P_{2}}{P_{1}}-\dfrac{M_{2}}{P_{2}}\dfrac{P_{2}}{P_{1}}-C_{1}  \right) + \beta\lambda_{2}\left( Y_{1}+\dfrac{B_{2}}{P_{2}}+\dfrac{M_{2}}{P_{2}}-C_{2}  \right) + \beta\delta_{2}\left( \dfrac{M_{2}}{P_{2}}-\theta C_{2}  \right)   $
\end{center}
Donde $ (1+i) $ es la tasa de inter�s nominal. 
\begin{itemize}
\item[a.] Suponga que una fracci�n de $ \theta $ del consumo en el periodo 2 se realiza con dinero, el resto se realiza a trav�s de transferencias electr�nicas. Obtenga las condiciones de primer orden.
\item[b.] Suponga que la empresa que da el servicio electronico deja de funcionar, con eso $ \theta=1 $. Analice los efectos de este cambio en la inflaci�n y en el consumo.
\end{itemize}

\end{enumerate}
\end{document}