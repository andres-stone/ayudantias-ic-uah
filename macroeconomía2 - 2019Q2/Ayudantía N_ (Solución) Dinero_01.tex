\documentclass[11pt,a4paper]{article}
\usepackage[latin1]{inputenc}
\usepackage[spanish]{babel}
\usepackage{amsmath}
\usepackage{amsfonts}
\usepackage{amssymb}
\usepackage{graphicx}
\usepackage[left=3cm,right=3cm,top=4cm,bottom=3cm]{geometry}
\author{Andr�s N. Dur�n}
\title{Macroeconom�a II}
\date{Mayo del 2019}
\begin{document}
\maketitle
\textbf{1. Dinero} Supondremos que las familias decidir�n cuanto dinero demandar para el consumo presente y futuro. Luego el problema de las familias es:
\begin{equation}
\max_{c_{t},m_{t},b_{2}} U(c,m): \left[ logC_{1}+log\dfrac{M_{1}}{P_{1}}+\beta\left( logC_{2}+log\dfrac{M_{2}}{P_{2}} \right)  \right] 
\end{equation}
\begin{align*}
s.a&\\
c_{1}&=Y_{1}-\dfrac{b_{2}}{1+\bar{r_{1}}}-m_{1}\\
c_{2}&=Y_{2}+b_{2}-m_{2}+\dfrac{m_{1}}{1+\pi_{2}}
\end{align*}
donde $ m_{t}=\dfrac{M_{t}}{P_{t}} $, con $ M_{t} $ es la oferta nominal de dinero propuesta por el Banco Central y $ (1+\bar{r_{1}})=\dfrac{(1+r_{1})}{1+\pi_{2}} $.
\begin{itemize}
\item[a.]Suponga un mundo de dos periodos $ (T=2) $. Escriba el problema que enfrenta las familias, y encuentre las condiciones de primer orden, para la restricci�n de primer periodo utilice $ \lambda_{1} $ y para la segunda restricci�n utilice $ \lambda_{2} $. Adem�s, considere que $ b_{1}=b_{3}=0 $ y $ m_{0}=0 $\\

\textbf{Respuesta:} La funci�n lagrangeano que resolvemos es:
\begin{align*}
L&=logC_{1}+log\dfrac{M_{1}}{P_{1}}+\beta\left( logC_{2}+log\dfrac{M_{2}}{P_{2}} \right)+\lambda_{1}\left( Y_{1}-\dfrac{b_{2}}{1+\bar{r_{1}}}-m_{1}-c_{1}\right)\\
&+\lambda_{2}\beta\left( Y_{2}+b_{2}-m_{2}+\dfrac{m_{1}}{1+\pi_{2}}-c_{2} \right) 
\end{align*}
Utilizando: $ m_{1}=\dfrac{M_{1}}{P_{1}} $ y $m_{2}=\dfrac{M_{2}}{P_{2}}$ el lagrangeano nos queda as�:
\begin{align*}
L&=logC_{1}+logm_{1}+\beta\left( logC_{2}+logm_{2} \right)+\lambda_{1}\left( Y_{1}-\dfrac{b_{2}}{1+\bar{r_{1}}}-m_{1}-c_{1}\right)\\
&+\lambda_{2}\beta\left( Y_{2}+b_{2}-m_{2}+\dfrac{m_{1}}{1+\pi_{2}}-c_{2} \right) 
\end{align*}

Las condiciones de primer orden son:
\begin{align*}
\dfrac{\partial L}{\partial C_{1}}&= \dfrac{1}{C_{1}}-\lambda_{1}=0 &(1)\\ 
\dfrac{\partial L}{\partial C_{2}}&= \beta\dfrac{1}{C_{2}}-\lambda_{2}\beta=0 &(2)\\
\dfrac{\partial L}{\partial m_{1}}&= \dfrac{1}{m_{1}}-\lambda_{1}+\dfrac{\lambda_{2}\beta}{\pi_{2}+1}=0 &(3)\\
\dfrac{\partial L}{\partial m_{2}}&= \beta\dfrac{1}{m_{2}}-\lambda_{2}\beta=0&(4)\\ 
\dfrac{\partial L}{\partial b_{2}}&= \dfrac{-\lambda_{1}}{1+\bar{r_{1}}}+\lambda_{2}\beta=0 &(5)
\end{align*}
\item[b.]Demuestre que la demanda de saldos reales para el primer y segundo periodo son $ m_{1}=c_{1}\left[ \dfrac{1+r_{1}}{r_{1}}\right]  $ y $ m_{2}=c_{2} $ respectivamente. Interprete el como reacciona la demanda de saldos en el primer periodo ante un cambio en la tasa de inter�s nominal.\\

\textbf{Respuesta:} De la ecuaci�n $ (5): \dfrac{\lambda_{1}}{1+\bar{r_{1}}}=\lambda_{2}\beta  $ en la ecuaci�n $ (3):\dfrac{1}{m_{1}}-\lambda_{1}+\dfrac{\lambda_{2}\beta}{\pi_{2}+1}=0 $ podemos obtener lo siguiente:
\begin{align*}
\dfrac{1}{m_{1}}-\lambda_{1}+\dfrac{\lambda_{2}\beta}{\pi_{2}+1}&=0\\
\dfrac{1}{m_{1}}-\lambda_{1}+\lambda_{1}\dfrac{1}{1+\bar{r_{1}}}\dfrac{1}{\pi_{2}+1}&=0\\
\dfrac{1}{m_{1}}-\dfrac{1}{C_{1}}+\dfrac{1}{C_{1}}\left( \dfrac{1}{1+r}\right)  &=0\\
\dfrac{1}{m_{1}}-\dfrac{1}{C_{1}}+\left( 1- \dfrac{1}{1+r}\right) &=0\\
\dfrac{1}{m_{1}}&=\dfrac{1}{C_{1}}\left( \dfrac{r}{1+r}\right)\\
m_{1}&=C_{1}\left( \dfrac{1+r}{r} \right) 
\end{align*}

De la ecuaci�n $ (4):\dfrac{1}{m_{2}}=\lambda_{2} $ y de $ (2): \lambda_{2}=\dfrac{1}{C_{2}} $ obtenemos:
\begin{align*}
\dfrac{1}{m_{2}}&=\lambda_{2}\\
\dfrac{1}{m_{2}}&=\dfrac{1}{C_{2}}\\
c_{2}&=m_{2}
\end{align*}

Diferenciando la demanda de saldos reales del primer periodo respecto a $ r_{1}: $
\begin{center}
$ \dfrac{\partial m_{1}}{\partial r_{1}}=-\dfrac{C_{1}}{r_{1}^{2}}<0 $
\end{center}
Esto nos dice que si el Banco Central aplica una pol�tica monetaria contractiva aumentando la tasa de inter�s, entonces disminuye la demanda de saldos reales del primer periodo como tambi�n el consumo presente.
\item[c.]Utilizando $ r=\bar{r}+\pi_{2} $, reescriba la demanda de dinero del primer periodo en funci�n de la tasa de inter�s real y la tasa de inflaci�n esperada. C�mo reacciona la demanda ante variaciones en la inflaci�n esperada?. Ayuda: suponga que $\beta\left( 1+\bar{r}\right) =1$, y que el Banco Central fija $ M_{1} $ y decide $ M_{2} $. �C�mo se ajusta la neutralidad del dinero en el an�lisis anterior?\\

\textbf{Respuesta:} Reescribiendo la ecuaci�n tenemos que: $ m_{1}=c_{1}\left( \dfrac{1+\bar{r}+\pi_{2}}{\bar{r}+\pi_{2}} \right)  $. Diferenciando la nueva ecuaci�n con respecto a la inflaci�n esperada obtendremos:
\begin{center}
$ \dfrac{\partial m_{1}}{\partial \pi_{2}}=-\dfrac{C_{1}}{\left( \bar{r}+\pi_{2}\right) ^{2}}<0   $
\end{center}
Luego si $\beta\left( 1+\bar{r}\right) =1 \Rightarrow C_{1}=C_{2}=c$, por tanto si el Banco Central decide aumentar $ ;M_{2} $, tendresos:
\begin{center}
$ \bar{m_{2}}=\dfrac{M_{2}}{P_{2}}=c\Rightarrow\uparrow M_{2} = \bar{m_{2}}P_{2}\uparrow\Rightarrow \dfrac{\uparrow P_{2}}{P_{1}}=1+\pi_{2}\uparrow $
\end{center}
Note que cuando $ \triangle M_{2}  $ es equivalente con $ \triangle P_{2} $ por lo que se cumple la neutralidad del dinero en el largo plazo, luego la demanda de saldos del primer periodo:
\begin{center}
$ \uparrow\pi_{2}\Rightarrow\downarrow\left( \dfrac{1+\bar{r_{1}}+\pi_{2}\uparrow}{\bar{r_{1}}+\pi_{2}} \right)c=\downarrow m_{1}=\dfrac{\bar{M_{1}}}{\uparrow P_{1}}  $
\end{center}
\end{itemize}
\end{document}