\documentclass[10pt,a4paper]{article}
\usepackage[latin1]{inputenc}
\usepackage[spanish]{babel}
\usepackage{amsmath}
\usepackage{amsfonts}
\usepackage{amssymb}
\usepackage{graphicx}
\usepackage[left=4cm,right=3cm,top=4cm,bottom=3cm]{geometry}
\author{Dinero e Inflaci�n}
\title{Macroeconom�a II}
\date{17 de Agosto del 2019}
\begin{document}
\maketitle

\begin{enumerate}
\item \textbf{Cash in Advance:} Este modelo afirma la necesidad de dinero para el consumo intertemporal de las familia. Considere el modelo que enfrentan las familias:
\begin{align*}
\max_{{c_{t},m_{t+1}}}U(c)&=ln c_{1} + \beta ln c_{2}\\
\end{align*}
Sujeto a :
\begin{equation}
P_{t}c_{t}=P_{t}y_{t}-\dfrac{B_{t+1}}{(1+i)}+B_{t}+M_{t}-M_{t+1}
\end{equation}
\begin{equation}
c_{t+1}=\dfrac{M_{t+1}}{P_{t+1}}
\end{equation}
\begin{itemize}
\item[a.]Suponga un mundo de dos periodos $ (T=2) $. Escriba el problema que resuelven las familias. Ayuda: Suponga que $ b_{1}=b_{3}=0 $ y $ m_{3}=0 $
\item[b.]Escriba el Lagrangeano, utilizando $ \lambda_{1}$,  $\lambda_{2}$ y $\gamma_{2} $ a las restricciones de primer y segundo periodo y de la restricci�n CIA respectivamente. Encuentre las CPO.
\item[c.]Del modelo, demuestre la siguiente proposici�n: \textit{En el corto plazo un aumento en la oferta de dinero por parte del banco central no provoca un alza inmediata de los precios, si no que de la producci�n, sin embargo en el largo plazo los precios aumentaran para que los saldos reales se contraigan y se mantenga la neutralidad del dinero.} Adem�s, suponga que $\beta\left( 1+r\right) =1$:\\
\end{itemize}
\item Usando el siguiente modelo:
\begin{center}
$ L: \dfrac{c_{1}^{1-\sigma}}{1-\sigma}+\beta\dfrac{c_{2}^{1-\sigma}}{1-\sigma}+\lambda_{1}\left( y_{1} + \dfrac{b_{2}}{1+i}  + m_{1} - m_{2} - c_{1} \right) + \beta\lambda_{2}\left( y_{2} + b_{2} + m_{2} - c_{2}   \right) + \beta\delta_{2}\left( m_{2} -\theta c_{2}  \right)   $
\end{center}
Donde $ (1+i) $ es la tasa de inter�s nominal. 
\begin{itemize}
\item[a.] Suponga que una fracci�n de $ \theta $ del consumo en el periodo 2 se realiza con dinero, el resto se realiza a trav�s de transferencias electr�nicas. Obtenga las condiciones de primer orden.
\item[b.] Suponga que la empresa que da el servicio electronico deja de funcionar, con eso $ \theta=1 $. Analice los efectos de este cambio en la inflaci�n y en el consumo.
\end{itemize}
\end{enumerate}
\end{document}