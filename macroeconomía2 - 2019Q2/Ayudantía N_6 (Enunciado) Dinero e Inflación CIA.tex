\documentclass[10pt,a4paper]{article}
\usepackage[latin1]{inputenc}
\usepackage[spanish]{babel}
\usepackage{amsmath}
\usepackage{amsfonts}
\usepackage{amssymb}
\usepackage{graphicx}
\usepackage[left=4cm,right=3cm,top=4cm,bottom=3cm]{geometry}
\author{Macroeconom�a II}
\title{Dinero e Inflaci�n}
\date{1 de Junio del 2019}
\begin{document}
\maketitle

\begin{enumerate}

\item \textbf{Parcial 2} Las siguientes ecuaciones describen el equilibrio en un modelo de dos per�odos donde el consumidor tiene al dinero como parte de su funci�n de utilidad:
\begin{equation}
c_{2}=\beta\left( 1+r\right)c_{1} 
\end{equation}
\begin{equation}
m_{2}=c_{2}
\end{equation}
\begin{equation}
m_{1}=c_{1}\dfrac{1+i}{i}
\end{equation}
\begin{equation}
1+i=\left( 1+r\right) \left(1+\pi_{2} \right)  
\end{equation}
\begin{equation}
c_{1}+\dfrac{c_{2}}{1+r}=y_{1}+\dfrac{y_{2}}{1+r}-m_{1}\dfrac{i}{1+i}-\dfrac{m_{2}}{1+r}
\end{equation}


Adicionalmente, asuma que $ \beta\left( 1+ r \right) =1  $ y recuerde que $ m_{1}=\dfrac{M_{1}}{P_{1}} $
\begin{itemize}
\item[a.] Muestre que el consumo en ambos per�odos no depende ni del nivel de precios, ni de los saldos monetarios, ni de la inflaci�n.
\item[b.] El se�oreaje $S$ corresponde al ingreso real que percibe quien tiene el monopolio
de la creaci�n del dinero, en este caso, el banco central. Este se calcula como
$S = \dfrac{\left(M_{2}-M_{1} \right) }{P_{1}}$. Exprese $S$ en t�rminos de la inflaci�n $\pi_{2}$, la tasa de
inter�s real $r$ y el ingreso real de los hogares $y_{1}$ e $y_{2}$.
\item[c.] Asuma que los precios son completamente flexibles y existe informaci�n perfecta.
�Qu� ocurre con $P_{1}$, $P_{2}$ $\pi_{2}$ y $S$ si el banco central decide aumentar $M_{1}$?
\end{itemize}
\item \textbf{Cash in Advance:} Este modelo afirma la necesidad de dinero para el consumo intertemporal de las familia. Considere el modelo que enfrentan las familias:
\begin{align*}
\max_{{c_{t},m_{t+1}}}U(c)&=ln c_{1} + \beta ln c_{2}\\
\end{align*}
Sujeto a :
\begin{equation}
P_{t}c_{t}=P_{t}y_{t}-\dfrac{B_{t+1}}{(1+i)}+B_{t}+M_{t}-M_{t+1}
\end{equation}
\begin{equation}
c_{t+1}=\dfrac{M_{t+1}}{P_{t+1}}
\end{equation}
\begin{itemize}
\item[a.]Suponga un mundo de dos periodos $ (T=2) $. Escriba el problema que resuelven las familias. Ayuda: Suponga que $ b_{1}=b_{3}=0 $ y $ m_{3}=0 $
\item[b.]Escriba el Lagrangeano, utilizando $ \lambda_{1}$,  $\lambda_{2}$ y $\gamma_{2} $ a las restricciones de primer y segundo periodo y de la restricci�n CIA respectivamente. Encuentre las CPO.
\item[c.]Del modelo, demuestre la siguiente proposici�n: \textit{En el corto plazo un aumento en la oferta de dinero por parte del banco central  provoca un alza inmediata de los precios, y no de la producci�n, sin embargo en el largo plazo los precios aumentaran para que los saldos reales se contraigan y se mantenga la neutralidad del dinero.} Adem�s, suponga que $\beta\left( 1+r\right) =1$:\\
\end{itemize}

\end{enumerate}
\end{document}