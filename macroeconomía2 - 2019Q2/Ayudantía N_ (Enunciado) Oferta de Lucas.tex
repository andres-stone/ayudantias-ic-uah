\documentclass[10pt,a4paper]{article}
\usepackage[latin1]{inputenc}
\usepackage[spanish]{babel}
\usepackage{amsmath}
\usepackage{amsfonts}
\usepackage{amssymb}
\usepackage{graphicx}
\usepackage[left=4cm,right=3cm,top=4cm,bottom=3cm]{geometry}
\author{Andr�s N. Dur�n}
\title{Macroeconom�a II}
\date{Mayo del 2018}
\begin{document}
\maketitle
\textbf{1. Oferta de Lucas} Suponga que la econom�a puede ser caracterizada por el siguiente modelo:

Oferta individual: \begin{align*}
y_{i}=p_{i}-E(p)
\end{align*}

Demanda Agregada: \begin{align*}
y=m-p
\end{align*}
Suponga que todas las firmas son iguales y que $ Y=1 $ indicando pleno empleo, por tanto $ Ln(Y)=y=0. $
\begin{itemize}
\item[(a)] Demuestre que si supone informaci�n completa el dinero es neutral. Gr�fique la demanda y la oferta agregada de este caso.
\item[(b)] Demuestre que si supone informaci�n incompleta el dinero deja de ser neutral. Suponga dos casos sobre la formaci�n de expectativas (Gr�fique):
\begin{itemize}
\item[i.] Hay un aumento en la demanda de dinero pero los productores no se dan cuenta y siguen pensando que $ E(p)=m $.
\item[ii.] Hay un aumento en la demanda de dinero y los productores usan informaci�n para hacer sus expectativas de corto plazo de esta manera: $ E(p)=pm+(1-p)m' $ con $ p \in [0,1] $. Explique las diferencias con el caso anterior. 
\end{itemize}
\end{itemize}

\textbf{2. Oferta de Lucas} Suponga que la econom�a puede ser representada por la existencia de productores aislados entre ellos (islas). Las ecuaciones claves son las siguientes:\\

Funci�n de producci�n del productor:
\begin{align*}
Q_{i}=L_{i}
\end{align*}

Utilidad del productor:
\begin{align*}
u_{1}&=C_{1}-\dfrac{1}{\gamma}L_{i}^{Y}\\
s.&a\\
PC_{i}&=P_{i}Q_{i}
\end{align*}

Demanda del productor:
\begin{align*}
q_{i}^{d}=y+z_{i}-\eta(p_{i}-p)
\end{align*}

Donde la demanda depende negativamente de los precios relativos y $ z_{i} $ es un elemento ex�geno al mercado. Adem�s al productor individual es $ z_{i}~N(O,\sigma^{2}) $

Demanda agregada:
\begin{align*}
y=m-p
\end{align*}

\begin{itemize}
\item[(a)] Demuestre que si $ z_{i}=0 $ y con informaci�n sim�trica la pol�tica monetaria es neutral.
\item[(b)] Demuestre que si $ z_{i}=0 $ y con informaci�n asim�trica la politica monetaria es no neutral. Observaci�n: Suponga expectativas racionales.
\end{itemize}

\end{document}