\documentclass[10pt,a4paper]{article}
\usepackage[latin1]{inputenc}
\usepackage[spanish]{babel}
\usepackage{amsmath}
\usepackage{amsfonts}
\usepackage{amssymb}
\usepackage{graphicx}
\usepackage[left=4cm,right=3cm,top=4cm,bottom=3cm]{geometry}
\author{Consumo, Ahorro e Inversi�n}
\title{Macroeconom�a II}
\date{06 de Julio del 2018}
\begin{document}
\maketitle
\begin{enumerate}
\item \textbf{Consumo Intertemporal:} Suponga un modelo de dos periodos, donde las familias maximizan su utilidad dada sus circunstancias econ�micas. Suponga que la utilidad instant�nea de los agentes se descuenta por una tasa $ \beta=\dfrac{1}{1+\rho} $ donde $\rho \in \left[ 0,+\infty\right)  $. Sea el problema del consumidor:
\begin{equation}
\max_{C_{t},N_{t}} U:= \sum_{t=1}^{T}\beta^{t-1}\left(  \dfrac{C_{t}^{1-\theta}}{1-\theta}-\dfrac{N^{1+\gamma}}{1+\gamma} \right) 
\end{equation}
Sujeto a la siguiente restricci�n:
\begin{equation}
C_{t}+\dfrac{C_{t+1}}{1+r}=w_{t}N_{t}+\dfrac{w_{t+1}N_{t+1}}{1+r}
\end{equation}
\begin{itemize}
\item[(a)] Resuelva el problema para una utilidad instant�nea gen�rica, es decir, $ u(C_{t},N{t}) $. Interprete el euler de consumo y la oferta de trabajo.
\item[(b)] Tanto la tasa de inter�s como la tasa de descuento son iguales a $ 5\% $, �C�mo se compara el nivel de consumo en el per�odo 1 con el consumo esperado en el periodo 2? Adem�s sabe que en el primer periodo el ingreso es de $5$ y en el periodo 2 es de $10$. (\textbf{Hint:} Suponga que $  \theta \rightarrow 1 $)
\item[(c)] �Qu� ocurr� con el consumo en ambos periodos si la tasa de inter�s sube a $ 10\% $?�En qu� direcci�n van el efecto sustituci�n y el efecto ingreso?
\end{itemize}
\end{enumerate}
\end{document}