\documentclass[10pt,a4paper]{article}
\usepackage[latin1]{inputenc}
\usepackage[spanish]{babel}
\usepackage{amsmath}
\usepackage{amsfonts}
\usepackage{amssymb}
\usepackage{graphicx}
\usepackage[left=4cm,right=3cm,top=4cm,bottom=3cm]{geometry}
\author{Andr�s N. Dur�n Salgado}
\title{Macroeconom�a II}
\date{24 de Mayo del 2018}
\begin{document}
\maketitle
\textbf{1. Modelo de Ciclos Reales} Supongamos una econom�a muy simple que puede ser caracterizada por un modelo donde se trabaja solo en el primer per�odo. La producci�n de ese per�odo se consume en parte y se invierte el resto para producir bienes el segundo per�odo. Las familias maximizan la siguiente funci�n de utilidad, donde se trabaja solo en el primer periodo:
\begin{equation}
\max U = ln \left( c_{1}\right) - \dfrac{N^{2}}{2}+\beta ln \left( c_{2}\right) 
\end{equation}

La restricci�n presupuestaria de la familia es:
\begin{align*}
C_{1}&=w_{1}N_{1}+\Pi_{1,1}+\Pi_{1,2}+\dfrac{B_{2}}{R_{1}}-T_{1}\\
C_{2}&=\Pi_{2,1}+\Pi_{2,2}-B_{2}-T_{2}\\
\end{align*}

Donde $ \Pi_{1,i}, i=1,2$ son utilidades de las firmas productoras de bienes y  $ \Pi_{2,i}, i=1,2$ son utilidades de las firmas productoras de capital.


Por otra parte, las firmas productoras de bienes tienen las siguientes funciones de producci�n:
\begin{align*}
Y_{1}&=A_{1}N_{1} &\\
Y_{2}&=A_{2}K_{2} 
\end{align*}


En este caso las funciones de beneficio son:
\begin{align*}
\max \Pi_{1,1}:& Y_{1}-w_{1}N_{1} &\\
\max \Pi_{2,1}:& Y_{2}-z_{2}K_{2} 
\end{align*}

Para las firmas productoras de capital la funci�n de beneficio es la siguiente:
\begin{align*}
\max & \underbrace{-K_{2}}_{\Pi_{1,2}}+\dfrac{1}{R_{1}} \underbrace{\left( Z_{2}K_{2}\right) }_{\Pi_{2,2}}
\end{align*}

El Gobierno act�a de la siguiente forma:
\begin{align*}
G_{1}&=T_{1}-\dfrac{B_{2}}{R_{1}}\\
G_{2}&=T_{2}+B_{2}\\
\end{align*}

Por simplicidad $ G_{1}=G_{2}=0 $

El equilibrio de la econom�a.
\begin{align*}
Y_{1}&=c_{1}+K_{2}\\
Y_{2}&=c_{2}\\
\end{align*}
\begin{itemize}
\item[(a)] Analice los efectos sobre el consumo, el trabajo y la producci�n de un aumento de la tecnolog�a de $ A_{1} > A_{2} = 1 $.
\item[(b)] Suponga que queremos aumentar la prociclicidad del trabajo, entonces, nuestra funci�n de utilidad cambia:
\begin{align*}
U=ln\left( C_{1}-\dfrac{N_{1}^{1+\nu}}{1+\nu}\right) + \beta ln \left( C_{2}\right) 
\end{align*}
Analice los efectos sobre consumo, trabajo y producci�n cuando hay un choque tecnol�gico en $ A_{1} > A_{2} = 1 $ . Adem�s compare con inciso (a)
\end{itemize}


\end{document}