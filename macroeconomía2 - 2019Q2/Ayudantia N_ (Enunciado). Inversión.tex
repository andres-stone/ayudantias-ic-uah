\documentclass[10pt,a4paper]{article}
\usepackage[latin1]{inputenc}
\usepackage[spanish]{babel}
\usepackage{amsmath}
\usepackage{amsfonts}
\usepackage{amssymb}
\usepackage{graphicx}
\usepackage[left=4cm,right=3cm,top=4cm,bottom=3cm]{geometry}
\author{Andr�s N. Dur�n Salgado}
\title{Macroeconom�a II}
\date{Marzo, 2018}
\begin{document}
\maketitle

\textbf{1. Firmas Productoras:} Suponga una econom�a simple donde existe las firmas productoras de bienes y servicios maximizan sus beneficios sujetos a sus costos.  

\begin{equation}
\max_{K_{t},N_{t}}\pi:{P_{t}Y_{y}-w_{t}N_{t}-z_{t}K_{t}}
\end{equation}

\begin{itemize}
\item[a.] Suponga una econom�a neocl�sica definida por una funci�n Cobb-Douglas: $ F(A,K,N)=A_{t}K_{t}^{\alpha}N_{t}^{1-\alpha} $, donde $ \alpha \in (0,1) $. Resuelva el problema de la firma productora de bienes y servicios, definiendo a $ w_{t} $  como salario nominal, a $ z_{t} $ como el precio nominal del capital, adem�s, interprete las condiciones de optimalidad.
\item[b.] Recuerde que la firma productora de capitales, maximiza sus beneficios sujeto a la ley de movimientos del capital, la cual est� definida como: $ K_{t+1}=I_{t}+(1+\delta)K_{t} $, donde $ \delta \in [0,1] $ es la tasa de depreciaci�n del capital. Resuelva el problema de maximizaci�n e interprete la condici�n de optimalidad sabiendo que la firma productoria de capital resuelve el siguiente problema: $\left( Z_{1}K_{1}-K_{2}\right) + \left( \dfrac{1}{R}\right) Z_{2}K_{2}$
\end{itemize}

\textbf{2. Q de Tobin:} Las decisiones de inversi�n en capital de una empresa, estar�n condicionadas al ratio entre el valor de mercado de la empresa $z_{t}$ y el coste de reemplazo de este, $1 + r$. Consideremos nuevamente la firma productora de capital, pero ahora supondremos que est�s deberan asumir un coste sobre la acumulaci�n de capital, por tanto deber�n resolver el siguiente modelo de optimizaci�n:\\
\begin{equation}
\max_{k_{t+1}} \sum_{t=1}^{T} \left( \dfrac{1}{1+r}\right) ^{t-1}(z_{t}K_{t}-I_{t})-\varnothing(K_{t+1})\\
\end{equation}
$s.a.$
\begin{align*}
K_{t+1}&=I_{t}+(1-\delta)K_{t}\\
K_{1}&>0\\
\varnothing(K_{2})&=0\\
\delta&=1.0
\end{align*}

Donde $\varnothing(K_{t+1})=\dfrac{\gamma}{2}(K_{t+1}-K_{t})^{2}$ es el coste de capital. Suponga nuevamente un mundo de dos periodos. Resuelva el problema y encuentre la Q de Tobin, interprete.
\end{document}