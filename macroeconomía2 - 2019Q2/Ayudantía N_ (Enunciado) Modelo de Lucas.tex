\documentclass[10pt,a4paper]{article}
\usepackage[latin1]{inputenc}
\usepackage[spanish]{babel}
\usepackage{amsmath}
\usepackage{amsfonts}
\usepackage{amssymb}
\usepackage{graphicx}
\usepackage[left=4cm,right=3cm,top=4cm,bottom=3cm]{geometry}
\author{Andr�s N. Dur�n}
\title{Macroeconom�a II}
\date{Mayo del 2018}
\begin{document}
\maketitle
\textbf{1. Modelo de Lucas} Suponga que la econom�a puede ser caracterizada por el siguiente modelo:
\begin{align*}
$Oferta Individual$&: y_{i}=p_{i}-E(p)\\
$Demanda Agregada$&: y=m-p
\end{align*}

Suponga que todas las firmas son iguales y que $ Y=1 $, indicando pleno empleo, por tanto $ Ln(Y)=y=0. $
\begin{itemize}
\item[(a)] Demuestre que si supone informaci�n incompleta el dinero deja de ser neutral. Dado el caso en que hay un aumento en la demanda de dinero y los productores se dan cuenta por lo que piensan que $ E(p)=m$.
\item[(b)] Suponga que a la oferta de lucas se le agrega productividad, tal que, la nueva funci�n se representa de la siguiente manera:

\begin{align*}
y=\left( P-E\left( p\right) \right) +a
\end{align*}

Donde $ a $ representa la productividad.
\begin{itemize}
\item[i.] La demanda agregada sigue siendo la misma, muestre gr�ficamente el pleno empleo.
\item[ii.] Del inciso $ (i.) $ muestre los efectos pro-ciclicos cuando aumenta la productividad.
\end{itemize}
\end{itemize}

\textbf{2. Oferta de Lucas} Suponga que la econom�a chilena puede ser caracterizada por el siguiente modelo:
\begin{align*}
$Oferta Individual$&: y=p-E(m)\\
$Demanda Agregada$&: y=m-p
\end{align*}

Suponga que el banco central sube la oferta de dinero de $ m $ a $ m' $, y que las expectativas se reajustan de la siguiente manera:
\begin{align*}
t=1 E(m')&=\dfrac{1}{3}m'\\
t=2 E(m')&=\dfrac{2}{3}m'\\
t=3 E(m')&=m'\\
\end{align*}
Esto es, las expectativas se reajustan en un $ 30\% $ en cada per�odo debido a informaci�n imperfecta. Grafique la din�mica de ajuste de la econom�a despu�s del shock.

\end{document}