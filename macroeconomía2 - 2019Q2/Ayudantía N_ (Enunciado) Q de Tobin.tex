\documentclass[10pt,a4paper]{article}
\usepackage[latin1]{inputenc}
\usepackage[spanish]{babel}
\usepackage{amsmath}
\usepackage{amsfonts}
\usepackage{amssymb}
\usepackage{graphicx}
\usepackage[left=4cm,right=3cm,top=4cm,bottom=3cm]{geometry}
\author{Andr�s N. Dur�n}
\title{Macroeconom�a II}
\date{02 de Mayo del 2018}
\begin{document}
\maketitle
\textbf{1. Q de Tobin} Suponga que la firma que produce bienes de inversi�n enfrenta costos cuadr�ticos de ajuste $ \dfrac{\gamma}{2}\left(K_{2} -K_{1} \right)^{2}  $ el gobierno cobra un impuesto $ t $ y se maximiza la siguiente funci�n de beneficios.
\begin{align*}
\max z_{1}K_{1}-K_{2}-\dfrac{\gamma}{2}\left( K_{2}-K_{1}\right)^{2} + \left( \dfrac{1}{R} \right) z_{2}K_{2}\left( 1-t \right) 
\end{align*}
\begin{itemize}
\item[(a)] Obtenga la Q de Tobin y explique cuando la firma decide reducir o aumentar la inversi�n. Explique c�mo afecta en esta explicaci�n el valor de $ \gamma $.
\item[(b)] Suponiendo que $ \gamma =1 $, $ z_{1}=z_{2}=K_{1}=1 $ y $ R=1,01 $, explique qu� sucede con la inversi�n si el gobierno decide bajar el impuesto de $ 0,1 $ a la mitad.
\end{itemize}


\textbf{2. Equilibrio General (Parcial 2017):} Suponga que la econom�a puede reducirse al siguiente conjunto de expresiones:

\textbf{Hogares}
\begin{align*}
U(c,N)&:=ln\left( c\right) - \dfrac{N^{2}}{2}\\
s&a\\
C&=WN.T
\end{align*}
Donde $ T $ son los impuestos.

\textbf{Firmas} La funci�n de producci�n que depende solo del trabajo $ N $ y donde $ A $ es productividad.
\begin{align*}
Y&=AN^{\alpha}
\end{align*}
En este caso la firma enfrenta rendimientos decrecientes y por tanto $ \alpha=\dfrac{1}{3} $. La firma maximiza las utilidades:
\begin{align*}
\max & Y-WN
\end{align*}

\textbf{Gobierno} Por simplicidad
\begin{align*}
G&=T=0
\end{align*}

\begin{itemize}
\item[(a)] Obtenga el equilibrio de la econom�a.
\item[(b)] Analice los efectos sobre el consumo, el trabajo y la producci�n de un aumento de la tecnolog�a de $ A=1 $ a $ A=2 $.
\item[(c)] Analice los efectos sobre el consumo, el trabajo y la producci�n de un aumento de $ \alpha = \dfrac{2}{3} $.
\end{itemize}

\end{document}