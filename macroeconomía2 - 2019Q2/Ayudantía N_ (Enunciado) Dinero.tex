\documentclass[10pt,a4paper]{article}
\usepackage[latin1]{inputenc}
\usepackage[spanish]{babel}
\usepackage{amsmath}
\usepackage{amsfonts}
\usepackage{amssymb}
\usepackage{graphicx}
\usepackage[left=4cm,right=3cm,top=4cm,bottom=3cm]{geometry}
\author{Andr�s N. Dur�n}
\title{Macroeconom�a II}
\date{Mayo del 2018}
\begin{document}
\maketitle
\textbf{1. Dinero (Parcial 2017)} Las siguientes ecuaciones describen el equilibrio en un modelo de dos per�odos donde el consumidor tiene al dinero como parte de su funci�n de utilidad:
\begin{align*}
c_{2}&=\beta\left( 1+r\right)c_{1} & (1)\\
m_{2}&=c_{2} & (2)\\
m_{1}&=c_{1}\dfrac{1+i}{i} & (3)\\
1+i&=\left( 1+r\right) \left(1+\pi_{2} \right)  & (4)\\
c_{1}+\dfrac{c_{2}}{1+r}&=y_{1}+\dfrac{y_{2}}{1+r}-m_{1}\dfrac{i}{1+i}-\dfrac{m_{2}}{1+r} & (5)
\end{align*}

Adicionalmente, asuma que $ \beta\left( 1+ r \right) =1  $ y recuerde que $ m_{1}=\dfrac{M_{1}}{P_{1}} $
\begin{itemize}
\item[(a)] Muestre que el consumo en ambos per�odos no depende ni del nivel de precios, ni de los saldos monetarios, ni de la inflaci�n.
\item[(b)] El se�oreaje $S$ corresponde al ingreso real que percibe quien tiene el monopolio
de la creaci�n del dinero, en este caso, el banco central. Este se calcula como
$S = \dfrac{\left(M_{2}-M_{1} \right) }{P_{1}}$. Exprese $S$ en t�rminos de la inflaci�n $\pi_{2}$, la tasa de
inter�s real $r$ y el ingreso real de los hogares $y_{1}$ e $y_{2}$.
\item[(c)] Asuma que los precios son completamente flexibles y existe informaci�n perfecta.
�Qu� ocurre con $P_{1}$, $P_{2}$ $\pi_{2}$ y $S$ si el banco central decide aumentar $M_{1}$?
\item[(d)] Discuta brevemente si usted ve alguna contradicci�n entre los objetivos y el
financiamiento del banco central
\end{itemize}

\end{document}