\documentclass[10pt,a4paper]{article}
\usepackage[latin1]{inputenc}
\usepackage[spanish]{babel}
\usepackage{amsmath}
\usepackage{amsfonts}
\usepackage{amssymb}
\usepackage{graphicx}
\usepackage[left=4cm,right=3cm,top=4cm,bottom=3cm]{geometry}
\author{Andr�s N. Dur�n Salgado}
\title{Macroeconom�a II}
\date{10 de Abril del 2018}
\begin{document}
\maketitle

\textbf{1. Familias:} Suponga que una familia maximiza la siguiente funci�n de utilidad que depende del consumo $ C $ y $ N $ el trabajo:
\begin{align*}
U(C,N)=\dfrac{C^{1-\sigma}}{1-\sigma} - \dfrac{N^{2}}{2}
\end{align*}
Sujeto a la siguiente restricci�n, donde $ W $ es el salario real: $ C=WN $. Suponga que la funci�n de producci�n es: $ Y=AN $.
\begin{itemize}
\item[(a)] Obtenga la cantidad de consumo y trabajo que decide la familia.
\item[(b)] Explique porque en este caso el consumo depende de la tecnolog�a corriente y no de la riqueza como en el caso del modelo de dos per�odos ense�ado en clase. Suponga $ A=1 $ y luego $ A=2 $.
\end{itemize}

\textbf{2. Equilibrio de la Econom�a} Suponga una econom�a que puede reducirse al siguiente conjunto de expresiones.

\textbf{Hogares:}
\begin{align*}
U&=\dfrac{C^{1-\sigma}}{1-\sigma} - \dfrac{N^{2}}{2}\\
C&=WN-T
\end{align*}
Donde $ T $ son los impuestos.\\

\textbf{Firmas:} La funci�n de producci�n que depende solo del trabajo $ N $ y donde $ A $ es la tecnolog�a.
\begin{align*}
Y=AN
\end{align*}
La firma maximiza las utilidades:
\begin{align*}
\max Y-WN
\end{align*}

\textbf{Equilibrio de la econom�a}. $ G $ es el gasto del gobierno:
\begin{align*}
Y=C+G
\end{align*}
Suponga que la pol�tica fiscal es un porcentaje del PIB:
\begin{align*}
G=0,1Y
\end{align*}

\begin{itemize}
\item[(a)] Analice los efectos sobre el consumo, el trabajo y la producci�n de un aumente de la tecnolog�a de $ A=1 $ a $ A=2 $
\item[(b)] Analice los efectos sobre el consumo, trabajo y la producci�n de un aumento de gasto de gobierno $ G=0,1Y $ a $ G=0,2Y $
\end{itemize}

\end{document}