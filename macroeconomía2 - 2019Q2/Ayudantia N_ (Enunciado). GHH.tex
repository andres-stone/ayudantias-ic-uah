\documentclass[10pt,a4paper]{article}
\usepackage[latin1]{inputenc}
\usepackage[spanish]{babel}
\usepackage{amsmath}
\usepackage{amsfonts}
\usepackage{amssymb}
\usepackage{graphicx}
\usepackage[left=4cm,right=3cm,top=4cm,bottom=3cm]{geometry}
\author{Andr�s N. Dur�n Salgado}
\title{Macroeconom�a II}
\date{05 de Junio del 2018}
\begin{document}
\maketitle
\textbf{1. Modelo de Ciclos Reales} Supongamos una econom�a muy simple que puede ser caracterizada por un modelo donde se trabaja solo en el primer per�odo. La producci�n de ese per�odo se consume en parte y se invierte el resto para producir bienes el segundo per�odo. Las familias maximizan la siguiente funci�n de utilidad, donde se trabaja solo en el primer periodo:
\begin{equation}
\max_{C_{1},C_{2},N_{1}} U:=ln\left( C_{1}-\dfrac{N_{1}^{1+\nu}}{1+\nu}\right) + \beta ln \left( C_{2}\right) 
\end{equation}

La restricci�n presupuestaria de la familia es:
\begin{align*}
C_{1}&=w_{1}N_{1}+\Pi_{1,1}+\Pi_{1,2}+\dfrac{B_{2}}{R_{1}}-T_{1}\\
C_{2}&=\Pi_{2,1}+\Pi_{2,2}-B_{2}-T_{2}\\
\end{align*}

Donde $ \Pi_{1,i}, i=1,2$ son utilidades de las firmas productoras de bienes y  $ \Pi_{2,i}, i=1,2$ son utilidades de las firmas productoras de capital.


Por otra parte, las firmas productoras de bienes tienen las siguientes funciones de producci�n:
\begin{align*}
Y_{1}&=A_{1}N_{1} &\\
Y_{2}&=A_{2}K_{2} 
\end{align*}


En este caso las funciones de beneficio son:
\begin{align*}
\max_{N_{1}} \Pi_{1,1}:& Y_{1}-w_{1}N_{1} &\\
\max_{K_{2}} \Pi_{2,1}:& Y_{2}-z_{2}K_{2} 
\end{align*}

Para las firmas productoras de capital la funci�n de beneficio es la siguiente:
\begin{align*}
\max_{K_{t+1}}: & \underbrace{-K_{2}}_{\Pi_{1,2}}+\dfrac{1}{R_{1}} \underbrace{\left( Z_{2}K_{2}\right) }_{\Pi_{2,2}}
\end{align*}

El Gobierno act�a de la siguiente forma:
\begin{align*}
G_{1}&=T_{1}-\dfrac{B_{2}}{R_{1}}\\
G_{2}&=T_{2}+B_{2}\\
\end{align*}

Por simplicidad $ G_{1}=G_{2}=0 $

El equilibrio de la econom�a.
\begin{align*}
Y_{1}&=c_{1}+K_{2}\\
Y_{2}&=c_{2}\\
\end{align*}
\begin{itemize}
\item[(a)] Obtenga el equilibrio general. Una posible causa de la baja tasa de crecimiento de la econom�a chilena es la baja productividad observada en los ultimos a�os. Al respecto, usted, como gerente general debe explicar en una reuni�n de directorio en el banco Hacemos Tus Sue�os
POSIBLES (BHSP) los alcances de esta situaci�n para la demanda por cr�ditos de consumo y corporativos del banco BHSP.
\item[(b)] Usted usa como base de an�lisis el modelo anterior y soluciona el siguiente caso: $A_{1} < A_{2} = 1$. Explique los efectos del shock productivos en t�rminos de producci�n, tasa de inter�s, consumo, inversi�n, empleo, salarios, precio del capital y demanda de cr�ditos. Gr�fique y suponga $\beta = 0.98$.
\end{itemize}


\end{document}