\documentclass[10pt,a4paper]{article}
\usepackage[latin1]{inputenc}
\usepackage[spanish]{babel}
\usepackage{amsmath}
\usepackage{amsfonts}
\usepackage{amssymb}
\usepackage{graphicx}
\usepackage[left=4cm,right=3cm,top=4cm,bottom=3cm]{geometry}
\author{Andr�s N. Dur�n}
\title{Macroeconom�a II}
\date{2018}
\begin{document}
\maketitle
\textbf{Q de Tobin (Parcial 2017):} Suponga que la firma que produce bienes de inversi�n enfrenta costos cuadr�ticos de ajuste: $ \dfrac{\gamma}{2}\dfrac{\left( K_{2}-K_{1} \right) ^{2}}{K_1} $ y maximiza la siguiente funci�n de beneficios.
\begin{equation}
\max Z_{1}K_{1}-K_{2}-\dfrac{\gamma}{2}\dfrac{\left( K_{2}-K_{1} \right) ^{2}}{K_1}+ \left( \dfrac{1}{R} \right) Z_{2}K_{2}
\end{equation}
\begin{itemize}
\item[(a)] Obtenga la Q de Tobin y explique cuando la firma decide reducir o aumentar la inversi�n. Explique c�mo afecta en esta explicaci�n el valor de $ \gamma $.
\item[(b)] Suponiendo que el banco central baja la tasa de inter�s en 25pbs ($ 0.25\% $), estime los efectos porcentuales sobre la inversi�n si $ \gamma $ toma valores de $ 1 $, $ 2 $ y $ 10 $ respectivamente. Adem�s, suponga que $ Z_{2} $ no cambia.
\end{itemize}
\end{document}