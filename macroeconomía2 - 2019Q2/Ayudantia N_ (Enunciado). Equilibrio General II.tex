\documentclass[10pt,a4paper]{article}
\usepackage[latin1]{inputenc}
\usepackage[spanish]{babel}
\usepackage{amsmath}
\usepackage{amsfonts}
\usepackage{amssymb}
\usepackage{graphicx}
\usepackage[left=4cm,right=3cm,top=4cm,bottom=3cm]{geometry}
\author{Equilibrio General}
\title{Macroeconom�a II}
\date{08 de Junio del 2019}
\begin{document}
\maketitle
\begin{enumerate}
\item En el mercado del trabajo existen dos tipos de trabajadores; Trabajadores Formales e Informales, la principal diferencia entre estos agentes es que trabajadores formales pueden distribuir el consumo a trav�s de periodos, no as�, agentes en el mercado informal. Entonces el problema de cada agente esta dado por:
\begin{itemize}
\item Trabajador Formal
\begin{align*}
U(C,N)& = \frac{C_{1}^{F{1-\sigma}}}{1-\sigma} + \beta \frac{C_{2}^{F{1-\sigma}}}{1-\sigma}- \frac{N_{1}^{F2}}{2}+\frac{N_{2}^{F2}}{2}\\
&s.a\\
&C_{1} = w_{1}N_{1}^{F}+B_{2}+\pi_{1}-T_{1}\\
&C_{2} = w_{2}N_{2}^{F}+R\cdot B_{2}+\pi_{2}-T_{2}\\
\end{align*}
\item Trabajador Informal
\begin{align*}
U(C,N)& = ln C_{1} + \beta ln C_{2}- \frac{N_{1}^{I1+\nu}}{1+\nu}+\frac{N_{2}^{I1+\nu}}{1+\nu}\\
&s.a\\
&C_{1}=w_{1}N_{1}^{I}-T_{1}\\
&C_{2}=w_{2}N_{2}^{I}-T_{2}\\
\end{align*}
\item[a.] Encuentre la oferta de trabajo en cada periodo para cada agente.
\item[b.] Para los trabajadores formales encuentre el euler de consumo.
\end{itemize}

\item Suponga que la funci�n de producci�n de la econom�a incluye energ�a para ambos periodos: $ Y_{t}=N_{t}^{\alpha}\left( K_{t}E_{t} \right)^{1-\alpha}  $ con $ t=1,2$. Las firmas maximizan sus beneficios $ Y_{t}-w_{t}N-z_{t}K{t} $ con $t=1,2 $.

\item Note que las familias son due�as de las empresas productoras de bienes y de capital, as� reciben las utilidades de estas empresas llamadas dividendos. Adem�s, la tasa de depreciaci�n es uno, por lo tanto, el capital del pr�ximo periodo es igual a la inversi�n:
\begin{align*}
K_{2}=\left( 1-\delta\right)K_{1}+I_{1} 
\end{align*}
Y las firmas productoras de capital maximizan sus beneficios: $ z_{1}K_{1}-K_{2}+\dfrac{Z_{2}k_{2}}{R} $

\item Sistematizando las ecuaciones encontradas en los enunciados anteriores, las restricciones pertinentes e incorporando los siguientes conceptos en un modelo de equilibrio general:
\begin{itemize}
\item Gobierno, interviene en la econom�a de la siguiente manera:
\begin{align*}
G_{1}&=T_{1}+ \mu^{G_{1}}\\
G_{2}&=T_{2}+ \mu^{G_{2}}
\end{align*}
Donde $ \mu^{G_{1}} $ y $ \mu^{G_{2}} $ son shock de gobierno.
\item Las condiciones de equilibrio de la econom�a:
\begin{align*}
C_{1}&=Y_{1}-K_{2}-G_{1}\\
C_{2}&=Y_{2}-G_{2}
\end{align*}
\item Las tecnolog�as para ambos periodos siguen el siguiente sistema de estimaciones:
\begin{align*}
A_{1}&=\Omega A_{-1}+ \mu^{A_{1}}\\
A_{2}&=\Omega A_{1}+ \mu^{A_{2}}
\end{align*}
Donde $ \mu^{A_{1}} $ y $ \mu^{A_{2}} $ son shock productivos.
\item Condiciones de Agregaci�n:
\begin{align*}
C_{1}&=\gamma C_{1}^{F}+(1-\gamma )C_{1}^{I}\\
C_{2}&=\gamma C_{2}^{F}+(1-\gamma )C_{2}^{I}\\
N_{1}&=\gamma N_{1}^{F}+(1-\gamma )N_{1}^{I}\\
N_{2}&=\gamma N_{2}^{F}+(1-\gamma )N_{2}^{I}\\
\end{align*}
\end{itemize}

\begin{itemize}
\item[a.] 	Utilizando las ecuaciones sistematizadas en el punto anterior y suponiendo que el 99\% de los agentes son formales y adem�s deudores.  Analice el impacto que tiene un shock tecnol�gico positivo en el segundo per�odo. Haga especial �nfasis en la ecuaci�n de Euler, es decir, analice con profundidad el consumo intertemporal. 
\item[b.]  	Repita inciso (a), ante una ca�da de la energ�a en el segundo periodo.
\end{itemize}

\end{enumerate}
\end{document}