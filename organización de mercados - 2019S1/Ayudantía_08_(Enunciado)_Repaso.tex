\documentclass[10pt,a4paper]{article}
\usepackage[latin1]{inputenc}
\usepackage[spanish]{babel}
\usepackage{amsmath}
\usepackage{amsfonts}
\usepackage{amssymb}
\usepackage{graphicx}
\usepackage[left=4cm,right=3cm,top=4cm,bottom=3cm]{geometry}
\author{Organizaci�n de Mercados}
\title{Repaso}
\date{08 de Junio del 2019}
\begin{document}
\maketitle
\begin{enumerate}
\item  Suponga que hay dos fabricantes de tablas de snowboard, Burton y K2. Estas empresas
venden productos similares pero no id�nticos (bienes diferenciados). Suponga que la demanda de
Burton es $ q_{B} = 900 -2p_{B} +p_{K}$ y la demanda de  K2 es $q_{K} = 900 -2p_{K} + p_{B}$ donde $q_{B}$ es la
demanda de Burton, $q_{K} $ es la demanda de K2, $p_{B}$ el precio de Burton y $p_{K}$ el precio de K2. Asuma
que los costos marginales de producci�n son cero para ambas empresas. Finalmente, suponga que las
empresas compiten en precios.
\begin{itemize}
\item[a.]	Encuentre el equilibrio de Nash si las empresas toman decisiones al mismo tiempo.
\item[b.]	Realice un gr�fico con las funciones de mejor respuesta, y el equilibrio de Nash.
\item[c.]	Suponga que Burton lanza una campa�a de publicidad que aumenta la demanda por Burton y
no cambia la demanda de K2 (b�sicamente atrae a nuevos clientes al snowboard que antes se
dedicaban a otro deporte). Utilice el gr�fico anterior para representar la nueva situaci�n. �Qu�
pasar� con el precio que fija Burton y el precio que fija K2?
\end{itemize}

\item Un art�culo de la prensa reporta que las aerol�neas Latam y Aerol�neas Argentinas trataron de incrementar la tarifas en un 4\% para la ruta Santiago de Chile - Buenos Aires. Sin embargo, esta estrategia fall�
porque Sky Airlines no quiso igualar los mayores precios. Suponga que Sky Airlines es una aerol�nea
que debe pagar una tasa de inter�s mayor en el mercado de capitales y tiene problemas de liquidez en el
presente, �c�mo puede explicar este comportamiento de Sky Airlines? Utilice lo aprendido en el curso
para fundamentar su respuesta.

\item El modelo de Bertrand (competencia en precios con bienes homog�neos) concluye que, como consecuencia de la competencia en precios, las utilidades de las empresas son cero incluso cuando hay
solamente dos empresas. �Por qu� cree que este resultado, en general, no se observa en la pr�ctica?

\item Suponga que una empresa est� considerando invertir en una innovaci�n que le permite disminuir sus
costos marginales de producci�n. Asuma que la empresa puede patentar esta innovaci�n y por lo tanto
la innovaci�n no puede ser copiada por las otras empresas en la industria. Adem�s, suponga que esta
industria se puede aproximar por un duopolio que compite en precios con bienes diferenciados.
Si la empresa que considera introducir la innovaci�n ignora el efecto estrat�gico (piensa que la otra
empresa no va a cambiar sus precios si introduce la innovaci�n), �los beneficios de la innovaci�n que
la empresa calcula ser�an mayores o menores a los beneficios si tuviese en cuenta el efecto estrat�gico?
Plantee la situaci�n en forma gr�fica para fundamentar su respuesta.
\end{enumerate}
\end{document}
