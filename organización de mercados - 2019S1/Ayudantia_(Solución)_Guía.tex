\documentclass[10pt,a4paper]{article}
\usepackage[latin1]{inputenc}
\usepackage[spanish]{babel}
\usepackage{amsmath}
\usepackage{amsfonts}
\usepackage{amssymb}
\usepackage{graphicx}
\usepackage[left=4cm,right=3cm,top=4cm,bottom=3cm]{geometry}
\author{Teor�a de Juegos y Mercados Oligop�licos}
\title{Organizaci�n de Mercados}
\date{11 de Abril del 2019}
\begin{document}
\maketitle
\begin{itemize}
\item[1.] Suponga que la empresa Tyco International tiene el control completo sobre el mercado de ganchos de pl�stico. La demanda inversa de ganchos viene dada por $ P = 3 - \dfrac{Q}{16000} $. Supongamos, adem�s, que el costo marginal de fabricar ganchos es constante e igual a $\$ 1$.
\begin{itemize}
\item[a.] Calcule la cantidad y el precio que maximizan los beneficios del monopolista
en este mercado. Calcule tambi�en el beneficio del monopolista.
\begin{align*}
\max \pi^{M} &= PQ-C(Q) \Rightarrow IMg(Q)=CMg(Q)\\
& IMg(Q)=CMg(Q)\\
& 3 -\dfrac{Q}{8000} = 1\\
& Q^{M}=16000\\
& P^{*} = 2
\end{align*}
Luego los beneficios:
\begin{align*}
\pi^{M}&= P^{M}Q^{M}-C(Q^{M})\\
& = 2\cdot 16000 - 1\cdot 16000
& = 160000 u.m.
\end{align*}
\item[b.] Calcule la cantidad y el precio que se obtendr�a en el mercado si fuese competitivo. Calcule tambi�n el beneficio que obtendr�a la firma en este caso.
\begin{align*}
\max \pi^{M} &= PQ-C(Q) \Rightarrow P=CMg(Q)
\end{align*}
Entonces $ P =1 $es la oferta de mercado, luego por equilibrio
\begin{align*}
1 = 3 -\dfrac{Q^{*}}{16000} \Rightarrow Q_{*} = 32000 \Rightarrow P^{*} =1
\end{align*}
Luego los beneficios de competencia es:
\begin{align*}
\pi^{*}&= P^{*}Q^{*}-C(Q^{*})\\
& = 1\cdot 32000 - 1\cdot 32000
& = 0 u.m.
\end{align*}
\item[c.] Compare los resultados obtenidos en los dos puntos anteriores en un gr�fico y calcule la p�rdida en el excedente del consumidor debido al monopolio.
El excedente del consumidor est� dado por:
\begin{align*}
EC = \dfrac{a - P^{i}}{2} \cdot Q^{i}
\end{align*}
Para $ i=*,M $ donde $ a $ es la m�xima disposici�n a pagar que tiene el consumidor, en este caso $ a=3 $. Por tanto
\begin{align*}
EC^{*} &= \dfrac{3-1}{2} \cdot 32000 = 32000\\
EC^{M} &= \dfrac{3-2}{2} \cdot 16000 = 8000
\end{align*}
Luego la perdida del EC es:
\begin{align*}
PEC = EC^{*}-EC^{*} = 24000u.m
\end{align*}
\end{itemize}

\item[2.] Todo equilibrio que proviene de la eliminaci�n iterada de estrategias estrictamente dominadas es un equilibrio de Nash, por lo tanto, todo equilibrio de Nash podr�a ser encontrado mediante la eliminaci�n iterada de estrategias estrictamente dominadas.\vspace{0.2 cm}
\textbf{Respuesta:} Falso, pues el NE ser� igual al equilibrio por eliminaci�n iterada de estrategias estrictamente
dominadas (EIEED), si es que existen tales estrategias, pues sino no existieran, entonces estos equilibrios no coinciden.

\item[3.] Dos locales de m�sica en vivo, Amadeus y Bachata, comparten la misma
zona y cuentan con una clientela fiel, estimada en 100 personas por noche en el
caso de Amadeus y 50 en el caso de Bachata. Ambos locales se plantean contratar o no a un m�sico famoso, que atraera m�s clientela siempre y cuando el local de al lado no elija la misma estrategia. Concretamente, el local Amadeus puede contratar por una sola noche a la estrella del piano Nizalbe, y el local Bachata a la estrella de la canci�n Lizza. Si Amadeus contrata a Nizalbe y Bachata decide no contratar,
Amadeus tendr� 40 clientes m�s y Bachata perder� a 10 de los suyos. Asimismo,
si Bachata contrata a Lizza cuando Amadeus no ha contratado m�sica en vivo, el n�mero de sus clientes aumentar� en 50, pero Amadeus perder� 30. Finalmente, si ambos locales contratan a m�sicos famosos (esto es, Amadeus a Nizalbe y Bachata a Lizza), el n�mero de sus clientes aumentar� en 20 y 10 personas, respectivamente. El beneficio que deja cada cliente se puede estimar en 10 euros para Amadeus y en 20 euros para Bachata. Llamemos N y L a los precios en euros de contratar a Nizalbe y a Lizza, respectivamente, siendo 200 < L < 400 y 200 < N < 400. 
\begin{itemize}
\item[a.] Represente el juego anteriormente citado (en forma normal), discutiendo cuales ser�n los equilibrios de Nash, dependiendo de los valores N y L.
\begin{align*}
J &= \left\lbrace A,B \right\rbrace \\
S_{i} & = \left\lbrace C, NC \right\rbrace \\
\forall i \in J
\end{align*}
Definiendo la funci�n de pago como sigue: $ u^{i} (s_{i},s_{j})$ por tanto
\begin{align*}
u^{A}(C,C) &=  (100+20)\cdot 10 - N = 1200 - N\\
u^{A}(C,NC) &=  (100+40)\cdot 10 - N = 1400 - N\\
u^{A}(NC,C) &=  (100-30)\cdot 10= 700\\
u^{A}(NC,NC) &=  (100+0)\cdot 10 - N = 1000\\
\end{align*}
\begin{align*}
u^{B}(C,C) &=  (50+10)\cdot 20 - L = 1200 - N\\
u^{B}(C,NC) &=  (50-10)\cdot 20 - L =800\\
u^{B}(NC,C) &=  (50+50)\cdot 20 - L = 2000 - N\\
u^{B}(NC,NC) &=  (50+0)\cdot 20 - L = 1000\\
\end{align*}
Por tanto el juego escrito de forma normal ser�:
\begin{center}

\begin{tabular}{c c | c | c |}
\hline\\
A/B& & C & NC\\
\hline
 & C & 1200-N, 1200-L & 1400-N,800\\
\hline
  & NC & 700,2000-L & 1000,1000\\
  \hline
\end{tabular}
\end{center}
Note que $ N,L \in (200,400) $, entonces analizando las mejores decisiones, tendremos que el equilibrio de Nash ser�
\begin{align*}
NE = \left\lbrace C,C \right\rbrace , \forall L,N \in (200,400)
\end{align*}
\item[b.] Ambos locales se plantean a�adir a las opciones anteriores una nueva alternativa: la barra libre. Si s�lo uno de ellos opta por introducir la barra libre se quedar� con todos los clientes, tanto si hay m�sica en vivo en el otro local como si no. En cambio, si ambos introducen la barra libre, los clientes se mantendr�n en su local preferido. El coste de la barra libre es de 200 euros para Amadeus y 100 euros para Bachata. Encuentre el nuevo equilibrio de Nash. �Cambia el
equilibrio de Nash encontrado en el inciso anterior?
\begin{align*}
J &= \left\lbrace A,B \right\rbrace \\
S_{i} & = \left\lbrace C, NC, BL \right\rbrace \\
\forall i \in J
\end{align*}
Luego la matriz de pago ser�
\begin{center}

\begin{tabular}{c c | c | c | c |}
\hline\\
A/B& & C & NC& BL\\
\hline
 & C & 1200-N, 1200-L & 1400-N,800& -N,2900\\
\hline
  & NC & 700,2000-L & 1000,1000,& 0,2900\\
  \hline
  & BL & 1300-L & 1300,0 & 800,900\\
  \hline
\end{tabular}
\end{center}
Donde obtendremos el nuevo equilibrio de Nash:$ NE = \left\lbrace BL,BL \right\rbrace  $
\end{itemize}

\item[4.] Una industria productora de un cierto bien est� integrada por s�lo dos empresas, cuyas respectivas funciones de costos totales son $ C_{1}(q_{1}) = \frac{1}{4}q_{1}^{2}+10q_{1}+20 $ y $ C_{2}(q_{2}) = \frac{1}{3}q_{2}^{2}+8q_{2}+18 $. La demanda de mercado de este bien, est� caracterizada por la siguiente funci�n inversa $ P(Q)=30-Q $, donde $ Q=q_{1}+q_{2} $

\begin{itemize}
\item[a.] Suponga que el mercado duopolico que preseta este bien se caracteriza por su competencia a la Cournot. Determine el equilibrio de Nash bajo este contexto.
\item[b.] Determine el precio de equilibrio, y los beneficios de ambas empresas.
\end{itemize}

\end{itemize}
\end{document}