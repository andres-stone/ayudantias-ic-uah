\documentclass[10pt,a4paper]{article}
\usepackage[latin1]{inputenc}
\usepackage[spanish]{babel}
\usepackage{amsmath}
\usepackage{amsfonts}
\usepackage{amssymb}
\usepackage{graphicx}
\usepackage[left=4cm,right=3cm,top=4cm,bottom=3cm]{geometry}
\author{Teor�a de Juegos y Mercados Oligop�licos}
\title{Organizaci�n de Mercados}
\date{11 de Abril del 2019}
\begin{document}
\maketitle
\begin{itemize}
\item[1.] \textbf{Juego Simultaneo:}	Soprole y Nestl� quieren entrar al mercado de los productos l�cteos preparados en Tumbucut�, donde no hay actualmente empresas ofreciendo estos productos. Ambas firmas deben decidir cu�l ser� la capacidad productiva de su planta de procesamiento. Sus alternativas son una planta con Gran, Mediana o Baja capacidad. Los pagos que obtengan dependen de la estrategia que siga cada firma, as� si ambas construyen una planta de Gran capacidad sus pagos son iguales a 0, si ambas construyen una de Mediana capacidad obtendr�n 16 cada una y si ambas plantas tienen Baja capacidad obtendr�n beneficios iguales a 18 cada una; si una planta es Grande y la otra es de Mediana capacidad, la primera empresa ganara 12 y la segunda 8; si una empresa es Grande y la otra es Baja capacidad los pagos ser�n 12 y 9 respectivamente; finalmente, si una planta es de Mediana capacidad y la otra de Baja capacidad, sus ganancias ser�n de 20 y 15 respectivamente.
\begin{itemize}
\item[a.] Expresa este juego en forma normal y resuelva por eliminaci�n iterada de estrategias estrictamente dominadas. Hazlo paso a paso e interprete sus resultados.
\item[b.] Encuentre el �nico equilibrio de Nash de este juego �Cu�l es? Explique.
\end{itemize}

\item[2.] Dos estaciones de servicio, A y B, est�n inmersas en una guerra de precios. Cada participante tiene la opci�n de subir el precio (R) o de mantener el precio bajo (C). Decidir�n su estrategia de forma simult�nea. Si ambas eligen C, ambas sufrir�n una perdida de $ 100\$ $. Si una opta por R y la otra elige C. (i) la que elige R pierde muchos de sus clientes y gana $ 0\$ $ y (ii) la que elige C gana muchos clientes nuevos y gana $ 1.000\$ $. Si ambas deciden R, la guerra de precios termina y cada uno gana $ 500\$ $.
\begin{itemize}
\item[a.] Traza la matriz de ganancias de este juego.
\item[b.] � Tiene alguno de los jugadores una estrategia dominante?
\item[c.] � Cu�ntos equilibrios de Nash tiene este juego? Explique.
\end{itemize}

\item[3.]	\textbf{Competencia a lo Cournot} Suponga un mercado duop�lico donde las firmas venden un bien homog�neo. La demanda viene dada por $ P(Q)=2-2Q $, donde $ Q=q_{1}+q_{2} $ y los costos de las firmas son $ c_{i}(q_{i})=\dfrac{1}{4}q_{i} $ para $ i=1,2 $
\begin{itemize}
\item[a.] Encuentre el equilibrio de Nash si las firmas compiten fijando cantidades. Gr�fique las funciones de mejor respuesta.
\item[b.] Determine el precio y la cantidad de mercado, adem�s, de los beneficios de cada firma � Qu� deber�a ocurrir con el bienestar del consumidor en un mercado duop�lico?
\item[c.] Suponga que ahora la estructura de costos de las firmas est�n dadas por $ c_{1}(q_{1})=\dfrac{1}{4}q_{1} $ y $ c_{2}(q_{2})=\dfrac{1}{2}q_{2} $ � C�mo cambia el resultado anterior? �Por qu�?
\end{itemize}

\item[4.]	\textbf{Competencia a lo Bertrand} Suponga que las empresas $ 1 $ y $ 2 $ tienen costos medios y marginales constantes, pero que $ CMg_{1}=10 $ y $ CMg_{2}=8 $. La demanda de la producci�n de las empresas es de
\begin{align*}
Q(P)=500-20P
\end{align*}
\begin{itemize}
\item[a.] Si las empresas practican la competencia de Bertrand � Cu�l ser� el precio de mercado en un equilibrio de Nash?
\item[b.] � Cu�les ser�n los beneficios de cada empresa?
\end{itemize}
\end{itemize}
\end{document}