\documentclass[10pt,a4paper]{article}
\usepackage[latin1]{inputenc}
\usepackage[spanish]{babel}
\usepackage{amsmath}
\usepackage{amsfonts}
\usepackage{amssymb}
\usepackage{graphicx}
\usepackage[left=4cm,right=3cm,top=4cm,bottom=3cm]{geometry}
\author{Juegos Repetidos}
\title{Organizaci�n de Mercados}
\date{18 de Mayo del 2019}
\begin{document}
\maketitle
\begin{itemize}
\item[1.] Alfa y Beta son dos empresas que venden productos homog�neos y compiten en precios.
La demanda viene dada por $D(p) = 100 - p$ y Los consumidores comprar�n a la firma que vende al
menor precio. En el caso que ambas empresas vendan al mismo precio, asumiremos que la mitad de
los consumidores comprar�n a cada firma. Suponga que el costo marginal de cada firma es constante e
igual a \$2.
\begin{itemize}
\item[a.] Encuentre el equilibrio de Bertrand en precios si las firmas eligen precios en forma simult�nea y
juegan s�lo una vez.
\item[b.] Suponga que el juego se repite y se juega dos veces. Sea $0 < \delta < 1$, el factor de descuento de las
firmas. Encontrar el equilibrio del juego repetido.
\item[c.]  Suponga que el juego se repite infinitas veces. Proponga estrategia de gatillo para cada firma de
forma de mantener el precio colusivo (el precio de un monopolista) en cada per�odo, asumiendo
que cada firma vuelve al equilibrio de Bertrand si alguna firma se desv�a del precio colusivo.
Calcular el valor m�nimo de $\delta$ tal que el precio colusivo es el equilibrio de este juego.
\item[d.] Imagine ahora que Alfa y Beta producen galletas de navidad, un producto que se vende s�lo
durante los �ltimos 3 meses del a�o. Si definimos un periodo como un trimestre, las empresas
interactuan cada 4 periodos. Calcula el valor m�nimo de ? tal que el precio colusivo es el equilibrio
de este juego. �Es m�s dif�cil o m�s f�cil coludir si las empresas interactuan solo en un trimestre
del a�o?
\end{itemize}

\item[2.] Dos firmas venden un bien homog�neo. Dadas las caracter�sticas de la industria, el modelo
de Cournot es una buena descripci�n de la situaci�n estrat�gica en la industria. La demanda en la
industria es $P(q) = 1 - 0.5q$ donde q = q1 + q2, los costos marginales son iguales a cero y el factor
de descuento es $\delta$
\begin{itemize}
\item[a.]	Suponga que las firmas interact�an por una �nica vez. Encuentre el equilibrio de Nash del juego.
Calcule los beneficios de las firmas.
\item[b.]	Suponga que las firmas interact�an en forma infinita. Las firmas acuerdan vender la cantidad que vender�a un monopolista. Seg�n el acuerdo, cada firma vende la mitad de dicha cantidad. Calcule los beneficios de las firmas en cada per�odo, en caso de cumplirse el acuerdo.
\item[c.]	Para sostener el acuerdo colusivo, las firmas siguen estrategias de gatillo donde amenazan con fijar las cantidades de Cournot en caso de un desv�o. Suponga que la firmas se encuentran en el per�odo t y alguna firma se ha desviado en una etapa anterior. Verifique que, dado que la firma 2 sigue una estrategia de gatillo, seguir una estrategia de gatillo es una mejor respuesta para la firma 1.
\item[d.]	Suponga que la firmas se encuentran en el per�odo t y ambas firmas han cumplido el acuerdo
colusivo en TODAS las etapas anteriores. Encuentre el factor de descuento m�nimo tal que,
dado que la firma 2 sigue una estrategia de gatillo, seguir una estrategia de gatillo es una mejor
respuesta para la firma 1.
\end{itemize}
\end{itemize}
\end{document}