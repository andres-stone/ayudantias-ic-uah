\documentclass[10pt,a4paper]{article}
\usepackage[latin1]{inputenc}
\usepackage[spanish]{babel}
\usepackage{amsmath}
\usepackage{amsfonts}
\usepackage{amssymb}
\usepackage{graphicx}
\usepackage[left=4cm,right=3cm,top=4cm,bottom=3cm]{geometry}
\author{Organizaci�n de Mercados}
\title{Repaso}
\date{11 de Junio del 2019}
\begin{document}
\maketitle
\begin{enumerate}

\item Considere el siguiente juego secuencial con entrada en un mercado con demanda
lineal igual a $p(q) = 9 - q$ donde $q$ es la producci�n en la industria. Primero, el incumbente
(firma 1) elige si invierte o no en una nueva tecnolog�a. El costo de inversi�n es igual a $K$. Si el
incumbente invierte, su costo marginal es $c_{L} = 1$, y si NO invierte su costo marginal es $c_{H} = 6$.
Luego, el incumbente y el entrante (firma 2), cuyo costo marginal es igual a $c_{H} = 6$, compiten
en cantidades a la Cournot. Tenga en cuenta que el costo de entrada para el potencial entrante
es cero.
\begin{itemize}
\item[a.] Demuestre que si el incumbente invierte, en el equilibrio de Nash en el juego a la Cournot,
el entrante produce cero. � Cu�nto produce el incumbente si el entrante produce cero?
\item[b.] � Cu�les son las utilidades de las rmas si el incumbente no invierte?
\item[c.] Encuentre aquellos valores de $K$ tal que el incumbente elige invertir en la nueva tecnolog�a.
\end{itemize}

\item Dos empresas compiten en un mercado con bienes homogeneos seleccionando sus precios. La
demanda esta dada por $ q = 40 - \dfrac{4}{3}p $ y sus costes por $(qi) = 5_{qi}$
\begin{itemize}
\item[a.]	Encuentre el equilibrio de Nash y los beneficios de equilibrio, si las empresas eligen sus
precios de manera simultanea.
\end{itemize}
Suponga que las empresas logran diferenciar sus productos. Ahora, la demanda de la Empresa 1 est� dada por $ q_{1}(p_{1}, p_{2} )= 20 - \dfrac{4}{3}p_{1} + \dfrac{2}{3}p_{2} $, mientras que la demanda de la empresa 2 es $ q_{2}(p_{1}, p_{2} )= 20 - \dfrac{4}{3}p_{2} + \dfrac{2}{3}p_{1} $. Las firmas mantienen sus costes de producci�n y elifen sus precios de manera simultanea.
\begin{itemize}
\item[b.] 	Encuentre las funciones de reaccion de ambas empresas y el equilibrio de Nash asociado. Encuentre tambi�n los beneficios en equilibrio.
\end{itemize}
Suponga ahora que la Empresa 1 elige primero su precio y que la Empresa 2, despu�s de observar el precio de su rival, elige el suyo.
\begin{itemize}
\item[c.] 	Encuentre el equilibrio de Nash perfecto en subjuegos y los beneficios asociados al equilibrio.
\end{itemize}
\item Suponga que dos firmas que producen bienes sustitutos deciden fusionarse.� Cu�l
es el efecto esperado sobre los precios? Su respuesta ser�a distintas si las firmas producen
bienes complementarios. Fundamente su respuesta.


\item  En el modelo de Stackelberg con entrada SIEMPRE es benecioso para una rma
establecida alterar su estrategia para prevenir o desalentar la entrada de un rival en lugar de
acomodarse a la entrada del rival. Verdadero o falso. Justique.


\item Suponga que una empresa est� considerando invertir en una innovaci�n que le permite disminuir sus
costos marginales de producci�n. Asuma que la empresa puede patentar esta innovaci�n y por lo tanto
la innovaci�n no puede ser copiada por las otras empresas en la industria. Adem�s, suponga que esta
industria se puede aproximar por un duopolio que compite en precios con bienes diferenciados.
Si la empresa que considera introducir la innovaci�n ignora el efecto estrat�gico (piensa que la otra
empresa no va a cambiar sus precios si introduce la innovaci�n), �los beneficios de la innovaci�n que
la empresa calcula ser�an mayores o menores a los beneficios si tuviese en cuenta el efecto estrat�gico?
Plantee la situaci�n en forma gr�fica para fundamentar su respuesta.
\end{enumerate}
\end{document}
