\documentclass[10pt,a4paper]{article}
\usepackage[latin1]{inputenc}
\usepackage[spanish]{babel}
\usepackage{amsmath}
\usepackage{amsfonts}
\usepackage{amssymb}
\usepackage{graphicx}
\usepackage[left=4cm,right=3cm,top=4cm,bottom=3cm]{geometry}
\author{Competencia Perfecta y Monopolio}
\title{Organizaci�n de Mercados}
\date{6 de Abril del 2019}
\begin{document}
\maketitle
\begin{itemize}
\item[1.] La funci�n inversa de demanda del mercado de un bien $ q $ es $ P(q) = 12-q $. En el mercado existe un monopolista cuya funci�n de costos es $ C(q) = 5+4q $
\begin{itemize}
\item[a.] Encuentre la cantidad ofrecida por el monopolista, el precio de mercado y los beneficios correspondientes.
\item[b.] Defina y calcule el �ndice de Lerner de poder de mercado.
\end{itemize}

\item[2.] La demanda inversa de mercado es: $ p(q)= \max{\left\lbrace 40-3q,0\right\rbrace } $. Cuyo costos totales son de $c(q) = q^{2}$
\begin{itemize}
\item[a.] Escribe el problema del monopolista y calcule el precio y cantidad bajo competencia perfecta y monopolio.
\item[b.] Grafique el problema y calcule la p�rdida de bienestar.
\end{itemize}

\item[3.] La demanda inversa de mercado es: $ D(p) = 100 p^{-2} $. Cuyo costos totales son de $c(q) = 5q$
\begin{itemize}
\item[a.] Calcule la elasticidad de demanda.
\item[b.] Calcule el precio y cantidad bajo competencia perfecta y monopolio.
\end{itemize}

\item[4.] Supongamos que la fabricaci�n de computadores port�tiles es una industria perfectamente competitiva. La demanda en el mercado est� descrita por la seiguiente demanda inversa:
\begin{align*}
p(Q) = 120 - \dfrac{9}{50}Q
\end{align*}
Supongamos ademas que hay $ N=50 $ oferentes que tienen exactamente la misma estructura de costos de largo plazo descrita por la siguiente ecuaci�n:
\begin{align*}
CT(q_{i}) &= 100 + q_{i}^{2} +100q_{i}\\
\forall i = 1,2,\ldots ,50
\end{align*}
\begin{itemize}
\item[a.] Determine la cantidad producida de computadores que permite m�ximizar
beneficios a cada una de las firmas.

\textbf{Respuesta:} Cada firma mazimizar� beneficios si est� en su condici�n de largo plazo, para esto:
\begin{align*}
CMg(q_{i}) &= CMe(q_{i})\\
2q_{i} +10 &= \dfrac{100}{q_{i}}+q_{i}+10\\
q_{i}^{*} = 10
\end{align*}
Por tanto cada firma produce 10 unidades.
\item[b.] Determine la curva de oferta de la industria, y el precio-cantidad que la vac�a. [Hint: recuerde que son firmas id�nticas.]

\textbf{Respuesta:} Debido a que hay 50 firmas id�nticas y cada firma tiene una oferta igual a $ p=2q_{i}+10 $ podemos sumar horizontalmente, es decir
\begin{align*}
S(Q) &= \sum_{i=1}^{50}S_{i}q_{i}\\
&= \sum_{i=1}^{50}p\\
&= \sum_{i=1}^{50}2q_{i} + \sum_{i}^{50}10\\
50p &= 2Q + 50\cdot 10\\
p &= \dfrac{1}{25}Q +10
\end{align*}
Por tanto la oferta de la industria es $ S(Q) = \dfrac{1}{25}Q +10 $. KLuego el precio-cantidad que vac�a el mercado se obtiene cuando
\begin{align*}
D(Q)&=S(Q)\\
120 - \dfrac{9}{50}Q &=\dfrac{1}{25}Q+10\\
Q^{*} &= 500\\
P^{*} & = 30
\end{align*}
\item[c.] Determine el bienestar social bajo estas condiciones.

\textbf{Respuesta:} El bienestar social $ W $ se obtiene sumando las ganancias netas que obtienen los consumidores y productores al enfrentarse a las condiciones de equilibrio, es decir:
\begin{align*}
W^{*} & = EC^{*} + EP^{*}\\
& = \dfrac{(120-30)\cdot 500}{2}+\dfrac{(30-10)\cdot 500}{2}\\
& = 27500
\end{align*}
\item[d.] Suponga ahora que por condiciones de eficiencia INTEL se queda con la demanda del mercado, convirti�ndose as� en la firma monop�lica de computadores. Determine la cantidad monop�lica del mercado.

\textbf{Respuesta:} El monopolista maximiza beneficios cuando $ IMg(q^{M}) = CMg(q^{M}) $, por tanto
\begin{align*}
IMg(q^{M}) &= CMg(q^{M})\\
120 - \dfrac{9}{25}q^{M} &= 2q^{m}+10\\
q^{M} & \approx 47\\
p^{M} & \approx 112\\
\end{align*}
\item[e.] Compute el nuevo bienestar social y determine la perdida social debido a la existencia de monopolio.
\textbf{Respuesta:} El bienestar social en este caso es:
\begin{align*}
W^{M} & = EC^{M} + EP^{M}\\
& = \dfrac{(120-112)\cdot 47}{2}+(103-10)\cdot 47+ \dfrac{(103-10)\cdot 47}{2}\\
& = 2797
\end{align*}
Luego la perdida social
\begin{align*}
PS = \dfrac{(112-103)(50-47)}{2} =13,5
\end{align*}
\end{itemize}
\end{itemize}
\end{document}