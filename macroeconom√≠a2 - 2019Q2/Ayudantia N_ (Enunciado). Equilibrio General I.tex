\documentclass[10pt,a4paper]{article}
\usepackage[latin1]{inputenc}
\usepackage[spanish]{babel}
\usepackage{amsmath}
\usepackage{amsfonts}
\usepackage{amssymb}
\usepackage{graphicx}
\usepackage[left=4cm,right=3cm,top=4cm,bottom=3cm]{geometry}
\author{Andr�s N. Dur�n Salgado}
\title{Macroeconom�a II}
\date{Marzo, 2018}
\begin{document}
\maketitle

\textbf{Equilibrio General y Shock Tecnol�gico:} Sistematizando las ecuaciones encontradas en los enunciados anteriores, las restricciones pertinentes e incorporando los siguientes conceptos en un modelo de equilibrio general:
\begin{itemize}
\item[-] Gobierno, interviene en la econom�a de la siguiente manera:
\begin{align*}
G_{1}&=T_{1}+ \mu^{G_{1}}\\
G_{2}&=T_{2}+ \mu^{G_{2}}
\end{align*}
Donde $ \mu^{G_{1}} $ y $ \mu^{G_{2}} $ son shock de gobierno.
\item[-] Las condiciones de equilibrio de la econom�a:
\begin{align*}
C_{1}&=Y_{1}-K_{2}-G_{1}\\
C_{2}&=Y_{2}-G_{2}
\end{align*}
\item[-] Las tecnolog�as para ambos periodos siguen el siguiente sistema de estimaciones:
\begin{align*}
A_{1}&=\Omega A_{-1}+ \mu^{A_{1}}\\
A_{2}&=\Omega A_{1}+ \mu^{A_{2}}
\end{align*}
Donde $ \mu^{A_{1}} $ y $ \mu^{A_{2}} $ son shock productivos.
\item[-] Condiciones de Agregaci�n:
\begin{align*}
C_{1}&=\gamma C_{1}^{R}+(1-\gamma )C_{1}^{NR}\\
C_{2}&=\gamma C_{2}^{R}+(1-\gamma )C_{2}^{NR}\\
N_{1}&=\gamma N_{1}^{R}+(1-\gamma )N_{1}^{NR}\\
N_{2}&=\gamma N_{2}^{R}+(1-\gamma )N_{2}^{NR}\\
\end{align*}
\end{itemize}

Utilizando las ecuaciones sistematizadas en el punto anterior y suponiendo que el $ 99\% $ de las familias son Ricardianas y adem�s deudoras. Analice el impacto que tiene un shock tecnol�gico positivo en el segundo per�odo. Haga especial �nfasis en la ecuaci�n de Euler, es decir, analice con profundidad el consumo intertemporal.
\end{document}