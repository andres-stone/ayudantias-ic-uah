\documentclass[10pt,a4paper]{article}
\usepackage[latin1]{inputenc}
\usepackage[spanish]{babel}
\usepackage{amsmath}
\usepackage{amsfonts}
\usepackage{amssymb}
\usepackage{graphicx}
\usepackage[left=4cm,right=3cm,top=4cm,bottom=3cm]{geometry}
\author{Macroeconom�a II}
\title{Equilibrio General}
\date{13 de Mayo del 2013}
\begin{document}
\maketitle
\begin{itemize}
\item[1.] \textbf{Modelo de Ciclos Reales} Supongamos una econom�a muy simple que puede ser caracterizada por un modelo donde se trabaja solo en el primer per�odo. La producci�n de ese per�odo se consume en parte y se invierte el resto para producir bienes el segundo. Las familias maximizan la siguiente funci�n de utilidad, donde se trabaja solo en el primer periodo:
\begin{equation}
\max U = ln \left( C_{1}\right) - \dfrac{N^{2}}{2}+\beta ln \left( C_{2}\right)
\end{equation}

La restricci�n presupuestaria de la familia es:
\begin{align*}
(1+t_{1}) C_{1}&=W_{1}N_{1}+\Pi_{1,1}+\Pi_{1,2}+\dfrac{B_{2}}{R_{1}}-T_{1}\\
(1+t_{2})C_{2}&=\Pi_{2,1}+\Pi_{2,2}-B_{2}\\
\end{align*}

Donde $ \Pi_{i,1} $ con $ i=1,2 $ son las utilidades de las firmas productoras de bienes y servicios y $ \Pi_{i,2} $ con $ i=1,2 $ son las utilidades de las firmas productoras de capital.


Por otra parte, las firmas productoras de bienes tienen las siguientes funciones de producci�n:
\begin{align*}
Y_{1}&=A_{1}N_{1}\\
Y_{2}&=A_{i}K_{2}
\end{align*}

En ese caso las funciones de beneficio son:
\begin{align*}
\Pi_{1,1}&=A_{1}N_{1}-W_{1}N_{1}\\
\Pi_{2,1}&=A_{2}K_{2}-Z_{2}K_{2}
\end{align*}

Para las firmas productoras de capital la funci�n de beneficio es la siguiente:
\begin{align*}
\Pi_{2}&=\underbrace{-K_{2}}_{\Pi_{2,1}}+\dfrac{1}{R_{1}}(1+t_{3})\underbrace{\left( Z_{2}K_{2} \right)}_{\Pi_{2,2}}
\end{align*}

El Gobierno act�a de la siguiente forma:
\begin{align*}
G_{1}&=T_{1}-\dfrac{B_{2}}{R_{1}}\\
G_{2}&=T_{2}+B_{2}\\
\end{align*}

Por simplicidad $ G_{1}=G_{2}=0 $

\begin{itemize}
\item[a.] Obtenga el equilibrio de la econom�a.
\item[b.] Solucione el modelo para el siguiente caso:  $ A_{1} = A_{2} = \bar{A} $; $ t_{1} = 0, t_{2}, t_{3} =0 $
\item[c.] Resuela el siguiente caso y explique los efectos de sobre todas las variables de la econom�a cuando $ t_{1} > 0 $ y $ t_{-i}; \forall_{i}=1 $
\end{itemize}

\end{itemize}

\end{document}