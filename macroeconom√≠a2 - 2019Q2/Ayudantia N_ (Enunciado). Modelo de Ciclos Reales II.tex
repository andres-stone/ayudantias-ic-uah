\documentclass[10pt,a4paper]{article}
\usepackage[latin1]{inputenc}
\usepackage[spanish]{babel}
\usepackage{amsmath}
\usepackage{amsfonts}
\usepackage{amssymb}
\usepackage{graphicx}
\usepackage[left=4cm,right=3cm,top=4cm,bottom=3cm]{geometry}
\author{Andr�s N. Dur�n Salgado}
\title{Macroeconom�a II}
\date{2018}
\begin{document}
\maketitle
\textbf{1. Modelo de Ciclos Reales} Supongamos una econom�a muy simple que puede ser caracterizada por un modelo donde se trabaja solo en el primer per�odo. La producci�n de ese per�odo se consume en parte y se invierte el resto para producir bienes el segundo y tercer per�odo. Las familias maximizan la siguiente funci�n de utilidad, donde se trabaja solo en el primer periodo:
\begin{equation}
\max U = ln \left( c_{1}\right) - \dfrac{N^{2}}{2}+\beta ln \left( c_{2}\right) + \beta^{2} ln  \left( c_{3}\right)
\end{equation}

La restricci�n presupuestaria de la familia es:
\begin{align*}
c_{1}&=w_{1}N_{1}+\Pi_{1,1}+\Pi_{1,2}+\dfrac{B_{2}}{R_{1}}-T_{1}\\
c_{2}&=\Pi_{2,1}+\Pi_{2,2}-B_{2}+\dfrac{B_{3}}{R_{2}}-T_{2}\\
c_{3}&=\Pi_{3,1}+\Pi_{3,2}-B_{3}-T_{3}\\
\end{align*}

Donde $ \Pi_{1,i}, i=1,2,3$ son utilidades de las firmas productoras de bienes y  $ \Pi_{2,i}, i=1,2,3$ son utilidades de las firmas productoras de capital.


Por otra parte, las firmas productoras de bienes tienen las siguientes funciones de producci�n:
\begin{align*}
Y_{1}&=A_{1}N_{1} &\\
Y_{i}&=A_{i}K_{2} &i=2,3
\end{align*}


En este caso las funciones de beneficio son:
\begin{align*}
\max \Pi_{1,1}:& Y_{1}-w_{1}N_{1} &\\
\max \Pi_{2,1}:& Y_{2}-z_{2}K_{2} & i=2,3\\
\end{align*}

Para las firmas productoras de capital la funci�n de beneficio es la siguiente:
\begin{align*}
\max & \underbrace{-K_{2}}_{\Pi_{1,2}}+\dfrac{1}{R_{1}} \underbrace{\left( Z_{2}K_{2}\right) }_{\Pi_{2,2}}+\dfrac{1}{R_{1}R_{2}} \underbrace{\left( Z_{3}K_{2}\right) }_{\Pi_{3,2}}
\end{align*}

El Gobierno act�a de la siguiente forma:
\begin{align*}
G_{1}&=T_{1}-\dfrac{B_{2}}{R_{1}}\\
G_{2}&=T_{2}+B_{2}-\dfrac{B_{3}}{R_{2}}\\
G_{3}&=T_{3}+B_{3}
\end{align*}

Por simplicidad $ G_{1}=G_{2}=G_{3}=0 $

El equilibrio de la econom�a.
\begin{align*}
Y_{1}&=c_{1}+K_{2}\\
Y_{2}&=c_{2}\\
Y_{3}&=c_{3}
\end{align*}
\begin{itemize}
\item[(a)] Analice los efectos sobre el consumo, el trabajo y la producci�n de un aumento de la tecnolog�a de $ A_{1}= A_{2} = A_{3} = A $.
\end{itemize}


\end{document}