\documentclass[10pt,a4paper]{article}
\usepackage[latin1]{inputenc}
\usepackage[spanish]{babel}
\usepackage{amsmath}
\usepackage{amsfonts}
\usepackage{amssymb}
\usepackage{graphicx}
\usepackage[left=4cm,right=3cm,top=4cm,bottom=3cm]{geometry}
\author{Andr�s N. Dur�n Salgado}
\title{Macroeconom�a II}
\date{22 de Marzo del 2018}
\begin{document}
\maketitle
\textbf{1. Consumo intertemporal:} Suponga un modelo de dos periodos, donde las familias maximizan su utilidad dada sus circunstancias econ�micas. Suponga que la utilidad instant�nea de los agentes se descuenta por una tasa $ \beta=\dfrac{1}{1+\rho} $ donde $\rho \in \left[ 0,+\infty\right)  $. Sea el problema del consumidor:
\begin{equation}
\max_{C_{t},N_{t}} U:= \sum_{t=1}^{T}\beta^{t-1}\left(  \dfrac{C_{t}^{1-\theta}}{1-\theta}-\dfrac{N^{1+\gamma}}{1+\gamma} \right) 
\end{equation}
Sujeto a la siguiente restricci�n:
\begin{equation}
C_{t}+\dfrac{C_{t+1}}{1+r}=w_{t}N_{t}+\dfrac{w_{t+1}N_{t+1}}{1+r}
\end{equation}
\begin{itemize}
\item[(a)] Resuelva el problema para una utilidad instant�nea gen�rica, es decir, $ u(C_{t},N{t}) $. Interprete el euler de consumo y la oferta de trabajo.
\item[(b)] Tanto la tasa de inter�s como la tasa de descuento son iguales a $ 5\% $, �C�mo se compara el nivel de consumo en el per�odo 1 con el consumo esperado en el periodo 2? Adem�s sabe que en el primer periodo el ingreso es de $5$ y en el periodo 2 es de $10$. (\textbf{Hint:} Suponga que $  \theta^{*}=1.0  $)
\item[(c)] �Qu� ocurr� con el consumo en ambos periodos si la tasa de inter�s sube a $ 10\% $?�En qu� direcci�n van el efecto sustituci�n y el efecto ingreso?
\end{itemize}

\textbf{2. M�s consumo intertemporal:} Considere una persona que vive dos
periodos, $t$ y $t + 1$, y sus ingresos son de 100 y 150 respectivamente. Si
la tasa de inter�s es del $15\%$:
\begin{itemize}
\item[(a)]Determine la restricci�n presupuestaria de este individuo y graf�quela.
\item[(b)]Suponga que a esta persona le interesa tener el mismo consumo en
ambos periodos. Encuentre el valor de este.
\item[(c)]Si las preferencias de este individuo son tales que desea consumir el
doble del primer per�odo t en el per�odo $t+1$, identifique el consumo en $t$ y $t + 1$.
\item[(d)]Explique conceptual y matem�ticamente que ocurre con el consumo de cada per�odo si la tasa de inter�s aumenta a $20\%$. Las preferencias de consumo del individuo se mantienen como en la parte (c).
\item[(e)]Identifique en un mismo gr�fico los resultados obtenidos en las partes (c) y (d), y explique los cambios ocurridos en el consumo debido a las variaciones de la tasa de inter�s.
\item[(f)]Suponga ahora que el gobierno ha instaurado un nuevo impuesto de suma alzada de 50 en cada per�odo. Encuentre la nueva restricci�n presupuestaria considerando una tasa del $15\%$ y grafique.
\item[(g)]Si la estructura de impuesto se mantiene de igual forma y el individuo desea consumir 40 en el primer per�odo:
\begin{itemize}
\item[i.]�Cu�l es el consumo en $ t+1 $?
\item[ii.]�C�mo cambia la recta presupuestaria si los impuestos cambian de estructura y se cobra 60 en $ t $ y 40 en $ t+1 $?
\item[iii.]�C�mo cambia el consumo en ambos per�odos?
\end{itemize}
\end{itemize}
\end{document}