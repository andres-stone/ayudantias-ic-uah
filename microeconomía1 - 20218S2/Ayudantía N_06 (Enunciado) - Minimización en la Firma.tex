\documentclass[10pt,a4paper]{article}
\usepackage[latin1]{inputenc}
\usepackage[spanish]{babel}
\usepackage{amsmath}
\usepackage{amsfonts}
\usepackage{amssymb}
\usepackage{graphicx}
\usepackage[left=4cm,right=3cm,top=4cm,bottom=3cm]{geometry}
\author{Andr�s N. Dur�n\footnote{1.andres.ds@gmail.com}}
\title{Microconom�a II}
\date{18 de Agosto del 2018}
\begin{document}
\maketitle
1. Sea la siguiente funci�n de producci�n
\begin{align*}
Q=F(L,K)=20L^{1/4}K^{1/2} 
\end{align*}
\begin{itemize}
\item[a.] Muestre los rendimientos a escala que presenta la funci�n de producci�n anterior. Explique sus resultados suponiendo un aumento en todos los factores de $ \lambda > 1 $.
\item[b.] Plantee el problema de optimizaci�n que la firma deber� resolver si es que esta se encuentra en una condici�n de corto plazo, y determine la funci�n de demanda condicional del trabajo suponiendo que dicha firma posee un stock de $ K_{0} $ m�quinas y desea producir $ q_{0} $ unidades.

Si el costo unitario de la mano de obra es de $ w $ u.m. mientras que el precio de arrendar capital es igual a $ r $ u.m. determine la funci�n de costos totales en el corto plazo para esta firma.
\item[c.] Suponga ahora que la firma est� en condiciones de largo plazo. Determine las funciones de demanda condicionadas de los factores, y encuentre la funci�n de costos de largo plazo.

Luego asuma que el precio del trabajo es $ w=2 $ u.m.  y el precio del capital es $ r=4 $. Determine las cantidades �ptimas de demandas condicionales.

\end{itemize}
\end{document}