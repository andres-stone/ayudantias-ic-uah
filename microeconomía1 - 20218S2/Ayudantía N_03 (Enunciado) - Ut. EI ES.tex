\documentclass[10pt,a4paper]{article}
\usepackage[latin1]{inputenc}
\usepackage[spanish]{babel}
\usepackage{amsmath}
\usepackage{amsfonts}
\usepackage{amssymb}
\usepackage{graphicx}
\usepackage[left=4cm,right=3cm,top=4cm,bottom=3cm]{geometry}
\author{Andr�s N. Dur�n \footnote{1.andres.ds@gmail.com}}
\title{Microconom�a II}
\date{28 de Julio del 2018}
\begin{document}
\maketitle
\textbf{1.} Verdadero, Falso o Incierto. Un bien inferior esta caracterizado por el aumento en su demanda ante una disminucion en el ingreso del individuo. Fundamente.\\

\textbf{2.} Verdadero, Falso o Incierto. Suponiendo que Juan consume tres bienes, A, B y C, de modo que tiene las siguientes preferencias sobre el orden de su consumo: ?prefiere consumir primero B antes que A y C antes que B?, entonces podr�?amos asegurar que Juan prefiere consumir A antes que C. Fundamente.\\


\textbf{3.} Verdadero, Falso o Incierto. Claudia Romero consume dos bienes cotidianamente, el primer bien representa un 10\% de su utilidad total, mientras que el segundo representa el otro 90\%. Si consume en este periodo 2 unidades del bien 1 y 18 unidades del bien 2, y el precio del bien 1 son 8 unidades monetarias, mientras que el precio del bien 2 aument� a 2 respecto al periodo anterior, entonces, �deber�a consumir mas del bien 1 que del bien 2?\\

\textbf{4.} Verdadero, Falso o Incierto. El excedente del consumidor es la ganancia que percibe el consumidor al pagar un precio menor que el que esta dispuesto a pagar. Fundamente.\\

\textbf{5.} Suponga que un consumidor representativo tiene preferencias descritas por la siguiente funci�n de utilidad:
\begin{align*}
u(x_{1},x_{2})=x_{1}^{0.4}x_{2}^{0.6}
\end{align*}
suponga que los precios de los bienes est�n dados por $p_{1}$ y $p_{2}$, mientras que el ingreso disponible del consumidor es $m$.
\begin{itemize}
\item[a.] Plantee el problema del consumidor, y su soluci�n �ptima en un gr�fico.
\item[b.] Determine la condici�n de equilibrio y relaci�n �ptima de consumo. [Hint: $RMS=-p_{1}/p_{2}$]
\item[c.] Determine la demanda de los bienes $x_{1}$ y $x_{2}$.
\item[d.] Suponiendo que los precios y el ingreso son $p_{1}=400, p_{2}=600$ y $m$=10.000. Determine la canasta �ptima y llamela $A$.
\item[e.] Suponga que el precio del bien 1 disminuy� a 200, ceteris paribus, determine la nueva canasta y llamenala C.
\item[f.] Determine el ingreso compensado para que el consumidor alcance nuevamente la canasta A.
\item[g.] Determine la canasta �ptima para el nuevo ingreso (ingreso compensado) y el nuevo precio del bien 1, y llamela B.
\item[h.] Calcule el {\bf Efecto Sustituci�n} y el {\bf Efecto Ingreso} ante el cambio del precio del bien 1.
\end{itemize}

\end{document}