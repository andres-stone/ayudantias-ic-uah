\documentclass[10pt,a4paper]{article}
\usepackage[latin1]{inputenc}
\usepackage[spanish]{babel}
\usepackage{amsmath}
\usepackage{amsfonts}
\usepackage{amssymb}
\usepackage{graphicx}
\usepackage[left=4cm,right=3cm,top=4cm,bottom=3cm]{geometry}
\author{Andr�s N. Dur�n \footnote{1.andres.ds@gmail.com}}
\title{Microconom�a II}
\date{01 de Septiembre del 2018}
\begin{document}
\maketitle
1. Suponga que un consumidor representativo tiene preferencias descritas por la siguiente funci�n de utilidad:
\begin{align*}
u(x_{1},x_{2})=x_{1}^{0.5}x_{2}^{0.5}
\end{align*}
suponga que los precios de los bienes est�n dados por $p_{1}$ y $p_{2}$, mientras que el ingreso disponible del consumidor es $m$.
\begin{itemize}
\item[a.] Plantee el problema del consumidor, y su soluci�n �ptima en un gr�fico.
\item[b.] Determine la condici�n de equilibrio y relaci�n �ptima de consumo. [Hint: $RMS=-p_{1}/p_{2}$]
\item[c.] Determine la demanda de los bienes $x_{1}$ y $x_{2}$.
\item[d.] Suponiendo que los precios y el ingreso son $p_{1}$=8.000, $p_{2}=$2.000 y $m$=10.000. Determine la canasta �ptima y llamela $A$.
\item[e.] Suponga que el precio del bien 1 disminuy� a 4.000, ceteris paribus, determine la nueva canasta y llamenala C.
\item[f.] Determine el ingreso compensado para que el consumidor alcance nuevamente la canasta A.
\item[g.] Determine la canasta �ptima para el nuevo ingreso (ingreso compensado) y el nuevo precio del bien 1, y llamela B.
\item[h.] Calcule el {\bf Efecto Sustituci�n} y el {\bf Efecto Ingreso} ante el cambio del precio del bien 1.
\end{itemize}

\newpage
2. Sea la siguiente funci�n de producci�n
\begin{align*}
Q=F(L,K)=AL^{1/2}K^{3/2} 
\end{align*}
\begin{itemize}
\item[a.] Muestre los rendimientos a escala que presenta la funci�n de producci�n anterior. Explique sus resultados suponiendo un aumento en todos los factores de $ \lambda > 1 $.
\item[b.] Plantee el problema de optimizaci�n que la firma deber� resolver si es que esta se encuentra en una condici�n de corto plazo, y determine la funci�n de demanda condicional del trabajo suponiendo que dicha firma posee un stock de $ K_{0} $ m�quinas y desea producir $ q_{0} $ unidades. El progreso tecnol�gico es $ A=10 $.

Si el costo unitario de la mano de obra es de $ w $ u.m. mientras que el precio de arrendar capital es igual a $ r $ u.m. determine la funci�n de costos totales en el corto plazo para esta firma.
\item[c.] Suponga ahora que la firma est� en condiciones de largo plazo. Determine las funciones de demanda condicionadas de los factores, y encuentre la funci�n de costos de largo plazo.

Luego asuma que el precio del trabajo es $ w= $ u.m.  y el precio del capital es $ r= $. Determine las cantidades �ptimas de demandas condicionales.

\end{itemize}

\end{document}