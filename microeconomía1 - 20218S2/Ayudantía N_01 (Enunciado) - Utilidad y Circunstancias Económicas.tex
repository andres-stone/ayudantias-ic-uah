\documentclass[10pt,a4paper]{article}
\usepackage[latin1]{inputenc}
\usepackage[spanish]{babel}
\usepackage{amsmath}
\usepackage{amsfonts}
\usepackage{amssymb}
\usepackage{graphicx}
\usepackage[left=4cm,right=3cm,top=4cm,bottom=3cm]{geometry}
\author{Andr�s N. Dur�n}
\title{Microconom�a II}
\date{07 de Julio del 2018}
\begin{document}
\maketitle
\textbf{1.} A Nicol�s le gusta la comida, pero no el humo del tabaco. Cuanto m�s alimento tiene, m�s dispuesto est� a renunciar para conseguir una reducci�n dada el humo del tabaco. Trace las curvas de indiferencia de Nicolas sabiendo que los alimentos y el humo del tabaco son los dos �nicos bienes existentes.\\

\textbf{2.} Imagine que un individuo va a comer pizza, la cual tiene dos ingredientes: Queso y aceitunas (bien 1 y bien 2). Dibuje el mapa de curvas de indiferencia explicando en qu� direcci�n crece la satisfacci�n en los siguientes casos:
\begin{itemize}
\item[a.] Asuma que el queso le gusta mucho, mientras que las aceitunas le desagradan (le resta satisfacci�n).
\item[b.] Asuma que el queso le gusta mucho, mientras que las aceitunas le dan lo mismo (no suma ni resta satisfacci�n).
\end{itemize}

\textbf{3.} Sea un individuo que tiene un ingreso de $ \$ 10.000 $ y desa consumir dos bienes; chicha y empanadas, cuyo precio unitario es de $ \$ 500 $ y $ \$ 1000 $ respectivamente.
\begin{itemize}
\item[a.] Exprese la restricci�n presupuestaria en gen�rico y luego incorpore los valores mencionados.
\item[b.] Encuentre los precios relativos e interprete.
\item[c.] Grafique, utilice chicha en el eje horizontal y empanadas en el eje vertical.
\end{itemize}

\textbf{4.} Comente si la afirmaci�n es verdadera, falsa o incierta; por qu�
\begin{itemize}
\item[a.] Una familia pobre se gana el loto y reparte equitativamente el dinero entre todos sus integrantes. Luego de ello, lo esperable es que cada persona consuma m�s de todos los bienes que consum�a con anterioridad.
\item[b.] La elasticidad de demanda de un bien producido internamente en un pa�s, podr�a depender de apertura de la econom�a, es decir, cuan abierta est� a las importaciones.
\end{itemize}

\textbf{5.} Explique gr�fica y conceptualmente que es el excedente del consumidor.\\

\textbf{6.} Sea un individuo que enfrenta la siguiente condici�n $ \dfrac{UMg_{1}}{UMg_{2}} = \dfrac{P_{1}}{P_{2}} * X $ cuando se sabe que est� consumiendo una cantidad positiva, mayor que uno, de ambos bienes. Analice la situaci�n desde la perspectiva del equilibrio y se�ale qu� deber�a hacer el individuo (en t�rminos de su decisi�n de consumo) si es que es racional. Fundamente su respuesta.


\end{document}