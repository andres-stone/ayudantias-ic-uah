\documentclass[10pt,a4paper]{article}
\usepackage[latin1]{inputenc}
\usepackage[spanish]{babel}
\usepackage{amsmath}
\usepackage{amsfonts}
\usepackage{amssymb}
\usepackage{graphicx}
\usepackage[left=4cm,right=3cm,top=4cm,bottom=3cm]{geometry}
\author{Andr�s N. Dur�n}
\title{Microconom�a II}
\date{11 de Agosto del 2018}
\begin{document}
\maketitle
\textbf{1.} Comente si es verdadero falso o incierto y por qu�.
\begin{itemize}
\item[-] En competencia perfecta, la firma estar� mejor mientras cobre un precio mayor que su costo marginal.
\item[-] Partiendo de una situaci�n de equilibrio, si la raz�n entre el producto marginal del trabajo y el salario aumenta (ex�genamente, manteniendo el salario constante), lo racional es que se contrate menos trabajadores, ya que con menos empleados es posible producir lo mismo que antes del cambio descrito.
\end{itemize}

\textbf{2.} Suponga que la curva de costos totales de una empresa es $ CT(q)=6+q^{2} $. Por su parte, la demanda de mercado es $P=100-Q$. Asuma que el mercado es competitivo y que las firmas son id�nticas y que est�n produciendo en el punto de eficiencia de escala, ie, en el punto en que se minimizan los costos totales medios, $ CTME = CMg $. Denote por $ q $ a la cantidad deproducci�n de cada firma, $ Q $ la cantidad de producci�n de mercado, $ P $ el precio de mercado y por $ N $ el n�mero de empresas.
\begin{itemize}
\item[a.] Encuentre la cantidad que produce cada firma.
\item[b.] Determine el equilirbio de mercado.
\item[c.] Determine los beneficios de cada firma.
\item[d.] Ahora que sabe la cantidad que produce cada firma $ (q) $ y la que produce el mercado total $(Q)$ �Cu�ntas empresas hay en este mercado?
\item[e.] Determine la funci�n de oferta de mercado, y luego calcule el excedente del productor.
\end{itemize}


\end{document}