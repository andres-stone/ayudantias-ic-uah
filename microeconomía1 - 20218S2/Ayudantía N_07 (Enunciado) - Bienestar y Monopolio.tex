\documentclass[10pt,a4paper]{article}
\usepackage[latin1]{inputenc}
\usepackage[spanish]{babel}
\usepackage{amsmath}
\usepackage{amsfonts}
\usepackage{amssymb}
\usepackage{graphicx}
\usepackage[left=4cm,right=3cm,top=4cm,bottom=3cm]{geometry}
\author{Andr�s N. Dur�n\footnote{1.andres.ds@gmail.com}}
\title{Microconom�a II}
\date{25 de Agosto del 2018}
\begin{document}
\maketitle
1. Suponga que la curva de costos totales de una empresa es $ CT(Q) = 12+Q^{2} $. Por su parte, la demanda de mercado es $ P=200-2Q $. Asuma que el mercado es competitivo y que las firmas son id�nticas.
\begin{itemize}
\item[i.] Bajo el supuesto de competencia perfecta, determine el equilibrio de mercado. � Cu�l es el bienestar social?
\item[ii.] Suponiendo que el mercado se reduce a una firma que enfrenta toda la demanda. Encuentre el nuevo equilibrio bajo �stas condiciones.
\item[iii.] Si el bienestar social en presencia de monopolio est� determinado por la ecuaci�n (1):
\begin{equation}
\label{1}
W^{M}=\dfrac{\left( a + P^{M} - CMg(Q^{M}) - c \right)Q^{M}}{2}
\end{equation}
Determine la perdida social en este mercado por la presencia de un monopolio.
\end{itemize}

2. Supongamos que la fabricaci�n de computadores port�tiles es una industria perfectamente competitiva. La demanda en el mercado est� descrita por la siguiente demanda inversa
\begin{equation}
p=120-\dfrac{5}{50}Q
\end{equation}
Suponga adem�s que hay $ N=50 $ oferentes que tienen exactamente la misma estructura de costos de largo plazo descrita por la ecuaci�n (3):
\begin{equation}
CT(q_{i})=100+q_{i}^{2}+10q_{i}, \forall i = 1, 2, 3, \ldots, 50
\end{equation}
\begin{itemize}
\item[i.] Determine la cantidad producida de computadores que permite m�ximizar beneficios a cada una de las firmas.
\item[ii.] Determine la curva de oferta de la industria, y el precio-cantidad que la vac�a.
\item[iii.] Determine el bienestar social bajo estas condiciones.
\item[iv.] Suponga ahora que por condiciones de eficiencia Intel se queda con la demanda del mercado, convirti�ndose as� en la firma monop�lica de computadores. Determine la cantidad mon�polica del mercado.
\item[v.] Compute el nuevo bienestar social y determine la perdida social debido a la existencia del monopolio.
\end{itemize}
\end{document}