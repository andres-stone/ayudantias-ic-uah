\documentclass[10pt,a4paper]{article}
\usepackage[latin1]{inputenc}
\usepackage[spanish]{babel}
\usepackage{amsmath}
\usepackage{amsfonts}
\usepackage{amssymb}
\usepackage{graphicx}
\usepackage[left=4cm,right=3cm,top=4cm,bottom=3cm]{geometry}
\author{Andr�s N. Dur�n}
\title{Microconom�a II}
\date{18 de Agosto del 2018}
\begin{document}
\maketitle
\textbf{1.} Supongamos que la fabricaci�n de computadores port�tiles es una industria perfectamente competitiva. La demanda en el mercado est� descrita por la siguiente demanda inversa
\begin{align*}
P=120-\dfrac{9}{50}Q
\end{align*}

Suponga adem�s que hay $ N=50 $ oferentes que tiene exactamente la misma estructura de costos de largo plazo descrita por:
\begin{align*}
CT(q_{i})=100+q_{i}^{2}+10q_{i}, \forall i=1,2,...,50
\end{align*}
\begin{itemize}
\item[a.] Determine la cantidad producida de computadores que permite maximizar beneficios a cada una de las firmas.
\item[b.] Determine la curva de oferta de la industria, y el precio-cantidad que la vac�a. Recuerde que son firmas id�nticas.
\item[c.] Determine el bienestar social bajo estas condiciones.
\item[d.] Suponga ahora que por condiciones de eficiencia Intel se queda con la demanda del mercado, convirti�ndose as� en la firma monop�lica de computadores. Determine la cantidad mon�polica del mercado.
\item[e.] Compute el nuevo bienestar social y determine la perdida debido a la existencia de monopolio.
\end{itemize}
\end{document}