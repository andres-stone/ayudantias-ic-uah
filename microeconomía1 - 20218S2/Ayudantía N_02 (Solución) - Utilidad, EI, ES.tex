\documentclass[10pt,a4paper]{article}
\usepackage[latin1]{inputenc}
\usepackage[spanish]{babel}
\usepackage{amsmath}
\usepackage{amsfonts}
\usepackage{amssymb}
\usepackage{graphicx}
\usepackage[left=4cm,right=3cm,top=4cm,bottom=3cm]{geometry}
\author{Andr�s N. Dur�n}
\title{Microconom�a II}
\date{14 de Julio del 2018}
\begin{document}
\maketitle
\textbf{1.} Suponga la siguiente funci�n de utilidad:
\begin{align*}
u(x_{1},x_{2})=x_{1}^{0,3}x_{2}^{0,7}
\end{align*}
Suponga que los precios de los bienes est�n dados por $ p_{1} $ y $ p_{2} $ mientras que el ingreso disponible del consumidor es $ m $.
\begin{itemize}
\item[a.] Plantee el problema del consumidor, y su soluci�n �ptima en un gr�fico.\\
\textbf{Respuesta:} El problema del consumidor viene dado por:
\begin{align*}
\max_{x_{1},x_{2}} u(x_{1},x_{2})&=x_{1}^{0,3}x_{2}^{0,7}\\
s&.a\\
m&=p_{1}x_{1}+p_{2}x_{2}
\end{align*}
Gr�fico adjunto.
\item[b.] Determine la condici�n de equilibrio.\\
\textbf{Respuesta:} La condici�n de equilibrio viene dada por $ RMS = - \dfrac{p_{1}}{p_{2}} $, ahora sabemos que la $ RMS=-\left( \dfrac{\alpha}{1-\alpha}\right) \dfrac{x_{2}}{x_{1}} $, entonces, reemplazando tenemos:
\begin{align*}
RMS&=-\left( \dfrac{0.3}{0.7}\right) \dfrac{x_{2}}{x_{1}}
\end{align*}
\item[c.] Determine la relaci�n �ptima de consumo.\\
\textbf{Respuesta:} Igualando la \textit{RMS} a la pendiente de la recta presupuestaria, tendremos la relaci�n �ptima de consumo:
\begin{align*}
\left( \dfrac{0.3}{0.7}\right) \dfrac{x_{2}}{x_{1}}&=\dfrac{p_{1}}{p_{2}}\\
x_{2}&=\dfrac{p_{1}}{p_{2}}\left( \dfrac{0.7}{0.3}\right)x_{1}\\
\end{align*}
\item[d.] Determine la demanda de los bienes $ x_{1} $ y $ x_{2} $.\\
\textbf{Respuesta:} Evaluando la relaci�n �ptima de consumo en la restricci�n presupuestaria:
\begin{align*}
m&=p_{1}x_{1}+p_{2}x_{2}\\
m&=p_{1}x_{1}+p_{2} \left( \dfrac{p_{1}}{p_{2}}\dfrac{0.7}{0.3} x_{1} \right) \\
m&=p_{1}x_{1}\left( \dfrac{10}{3}\right) 
\end{align*}

Por tanto la demanda individual del bien $ x_{1} =\dfrac{3}{10}\dfrac{m}{p_{1}} $, luego, esta cantidad optima la reemplazamos en la relaci�n �ptima obtendremos la demanda individual del bien $ x_{2} = \dfrac{7}{10}\dfrac{m}{p_{2}} $
\item[e.] Suponiendo que los precios y el ingreso son $ p_{1}=30 $, $ p_{2}=40 $ y $ m=20.000 $. Determine la canasta y llamela $ A $.\\
\textbf{Respuesta:} Evaluando los valores en las curvas de demanda:
\begin{align*}
x_{1}&=\dfrac{3}{10}\dfrac{m}{p_{1}}=\dfrac{3}{10}\dfrac{20000}{30}=200\\
x_{2}&=\dfrac{7}{10}\dfrac{m}{p_{2}}=\dfrac{3}{10}\dfrac{20000}{40}=350
\end{align*}

Finalmente la canasta �ptima de consumo es: $ A=(x_{1}^{A},x_{2}^{A})=(200,350) $
\item[f.] Suponga que el precio del bien 1 aumento a $ p_{1}=40 $, ceteris paribus, determine la nueva canasta y llamela $ C $.\\
\textbf{Respuesta:} Evaluando los valores en las curvas de demanda:
\begin{align*}
x_{1}&=\dfrac{3}{10}\dfrac{m}{p_{1}^{'}}=\dfrac{3}{10}\dfrac{20000}{40}=150\\
x_{2}&=\dfrac{7}{10}\dfrac{m}{p_{2}}=\dfrac{3}{10}\dfrac{20000}{40}=350
\end{align*}

Finalmente la canasta �ptima de consumo es: $ C=(x_{1}^{C},x_{2}^{C})=(150,350) $
\item[g.] Determine el ingreso compensado para que el consumidor alcance nuevamente la canasta.\\
\textbf{Respuesta:} El ingreso disponible se obtiene a partir de la siguiente formula:
\begin{align*}
m^{'}&=(p_{1}^{'}-p_{1})x_{1}^{A}+m\\
m^{'}&=(40-30)200+20000\\
m^{'}&=22000\\
\end{align*}
\item[h.] Determine la canasta �ptima para que el nuevo ingreso (ingreso compensado) y el nuevo precio del bien 1, llamela B.
\begin{align*}
x_{1}&=\dfrac{3}{10}\dfrac{m^{'}}{p_{1}^{'}}=\dfrac{3}{10}\dfrac{22000}{40}=165\\
x_{2}&=\dfrac{7}{10}\dfrac{m^{'}}{p_{2}}=\dfrac{3}{10}\dfrac{22000}{40}=385
\end{align*}

Finalmente la canasta �ptima de consumo es: $ B=(x_{1}^{B},x_{2}^{B})=(165,385) $
\item[i.] Calcule el Efecto Sustituci�n y el Efecto Ingreso ante el cambio del precio del bien 1.\\
\textbf{Respuesta:} El Efecto Sustituci�n es el traslado desde $ x_{1}^{A}\rightarrow x_{1}^{B} $ y el Efecto Ingreso es el traslado desde $ x_{1}^{B}\rightarrow x_{1}^{C} $, entonces:
\begin{align*}
ES&=x_{1}^{A}-x_{1}^{B}=200-165=35\\
EI&=x_{1}^{B}-x_{1}^{C}=165-150=15\\
ET&=ES+EI=50
\end{align*}
\item[j.] Gr�fique las situaciones.\\
\textbf{Respuesta:} Gr�fico adjunto.
\end{itemize}
\end{document}