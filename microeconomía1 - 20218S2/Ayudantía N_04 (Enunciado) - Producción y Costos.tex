\documentclass[10pt,a4paper]{article}
\usepackage[latin1]{inputenc}
\usepackage[spanish]{babel}
\usepackage{amsmath}
\usepackage{amsfonts}
\usepackage{amssymb}
\usepackage{graphicx}
\usepackage[left=4cm,right=3cm,top=4cm,bottom=3cm]{geometry}
\author{Andr�s N. Dur�n}
\title{Microconom�a II}
\date{11 de Agosto del 2018}
\begin{document}
\maketitle
1. Sea la siguiente funci�n de producci�n: $ Q=F(L,K)=20L^{1/5}K^{2/5} $
\begin{itemize}
\item[a.] � Cu�nto \textit{output} se obtiene al introducir 100 unidades de trabajo y 100 de capital?
\item[b.] Obtenga la productividad media del trabajo.
\item[c.] Obtenga la productividad media del capital.
\item[d.] Obtenga la productividad marginal del trabajo.
\item[e.] Obtenga la productividad marginal del capital.
\item[f.] Gr�fique un mapa de curas de \textit{isocuantas} para un nivel de producto dado y encuentre la TMST de estas.
\end{itemize}

2. Sea la siguiente funci�n de producci�n
\begin{align*}
Q=F(L,K)=2L^{1/2}K^{1/2} 
\end{align*}
\begin{itemize}
\item[a.] Muestre los rendimientos a escala que presenta la funci�n de producci�n anterior. Explique sus resultados suponiendo un aumento en todos los factores de $ \lambda > 1 $.
\item[b.] Encuentre las curvas de costo de corto plazo (costo total, costo variable, costo fijo, costo medio, costo variable medio y costo fijo medio), considerando que el capital es 81, el salario es igual a 3 y el arriendo del capital es igual a 3.

Asuma que la funci�n de producci�n y el precio de los insumos se mantiene, sin embargo, ahora estamos en el largo plazo (ambos factores de producci�n son variables).
\item[c.] Encuentre la curva de costos de largo plazo; primero en gen�rico y luego reemplazando los precios expresado en (b.)
\end{itemize}
\end{document}