\documentclass[10pt,a4paper]{article}
\usepackage[latin1]{inputenc}
\usepackage[spanish]{babel}
\usepackage{amsmath}
\usepackage{amsfonts}
\usepackage{amssymb}
\usepackage{graphicx}
\usepackage[left=4cm,right=3cm,top=4cm,bottom=3cm]{geometry}
\author{Andr�s N. Dur�n}
\title{Microconom�a II}
\date{14 de Julio del 2018}
\begin{document}
\maketitle
\textbf{1.} Suponga la siguiente funci�n de utilidad:
\begin{align*}
u(x_{1},x_{2})=x_{1}^{0,3}x_{2}^{0,7}
\end{align*}

Suponga que los precios de los bienes est�n dados por $ p_{1} $ y $ p_{2} $ mientras que el ingreso disponible del consumidor es $ m $.
\begin{itemize}
\item[a.] Plantee el problema del consumidor, y su soluci�n �ptima en un gr�fico.
\item[b.] Determine la condici�n de equilibrio.
\item[c.] Determine la relaci�n �ptima de consumo.
\item[d.] Determine la demanda de los bienes $ x_{1} $ y $ x_{2} $.
\item[e.] Suponiendo que los precios y el ingreso son $ p_{1}=30 $, $ p_{2}=40 $ y $ m=20.000 $. Determine la canasta y llamela $ A $.
\item[f.] Suponga que el precio del bien 1 aumento a $ p_{1}=40 $, ceteris paribus, determine la nueva canasta y llamela $ C $
\item[g.] Determine el ingreso compensado para que el consumidor alcance nuevamente la canasta.
\item[h.] Determine la canasta �ptima para que el nuevo ingreso (ingreso compensado) y el nuevo precio del bien 1, llamela B.
\item[i.] Calcule el Efecto Sustituci�n y el Efecto Ingreso ante el cambio del precio del bien 1.
\item[j.] Gr�fique las situaciones.
\end{itemize}
\end{document}