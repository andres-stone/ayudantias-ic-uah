\documentclass[10pt,a4paper]{article}
\usepackage[latin1]{inputenc}
\usepackage[spanish]{babel}
\usepackage{amsmath}
\usepackage{amsfonts}
\usepackage{amssymb}
\usepackage{graphicx}
\usepackage[left=4cm,right=3cm,top=4cm,bottom=3cm]{geometry}
\author{Andr�s N. Dur�n}
\title{Econom�a Matem�tica}
\date{09 de Abril del 2018}
\begin{document}
\maketitle
\textbf{1.} Asumiendo que cada una de las siguientes funciones son lineales, interprete econ�micamente la pendiente de cada una de ellas: (i) $ G(x) $ es el costo para un consumidor de comprar $x$ unidades de un bien, (ii) $ C(Y) $ es el consumo de un pa�s cuando su ingreso nacional es $ Y $, y (iii) $ F(q) $ es el ingreso de una empresa que produce $ q $ unidades de un bien.\\

\textbf{2.} Represent	e gr�ficamente en un plano cartesiano los siguientes conjuntos: (i) $ A:=\left\lbrace \left(x,y \right): 2x+4y \geq 5  \right\rbrace  $, (ii) $ A:=\left\lbrace \left(x,y \right): x-3y+2 \geq 0  \right\rbrace  $, y (i) $ A:=\left\lbrace \left(x,y \right): x-y \geq 1  \right\rbrace  $.\\

\textbf{3.} Suponga que una empresa puede producir hasta $ 100 $ unidades de un bien; es decir $ 0 \leq Q \leq 100 $. El costo total de producir $ Q$ unidades es tambi�n una funci�n lineal con pendiente igual a 4 y coeficiente de posici�n igual a cero. El ingreso total de producir $ Q $ unidades es tambi�n una funci�n lineal pero con pendiente igual a $ a > 0 $ y coeficiente de posici�n igual a cero. Asuma que la empresa desea maximizar sus beneficios.
\begin{itemize}
\item[(a)] Obtenga una f�rmula para los beneficios totales y encuentre la cantidad $ Q $ que maximiza los beneficios de la empresa si (i) $ a=3 $, (ii) $ a=4 $ y si (iii) $ a=5 $.
\item[(b)] Suponga que $ a=3 $ y que la empresa recibe una donaci�n igual a $ \$ $ $ b  >0 $ independientemente de cuantas unidades produzca. Encuentre una expresi�n que represente los ingresos totales de la empresa incluyendo la donaci�n $ b $. Encuentre la cantidad que maximiza los beneficios de la presa si $ b=100 $.
\item[(c)] Suponga que $ a=3 $ pero que ahora la empresa recibe una donaci�n igual a $ b=100 $ pesos si y solo si producen al menos $ 50 $ unidades; es decir $ b=100 \Leftrightarrow Q\geq 50$ . Represente matem�ticamente y gr�ficamente la funci�n de ingresos de la empresa.
\item[(d)] Suponga que $ a=3 $ y que $ b=60 $ pesos si y solo si producen al menos 50 unidades. Encuentre la cantidad $ Q $ que maximiza los beneficios de la empresa.
\end{itemize}

\textbf{4.} La funci�n de demanda y oferta de un bien est�n dadas respectivamente por $ Q_{D}=100-P $ y $ Q_{S}=4P $
\begin{itemize}
\item[(a)]Obtenga las funciones de demanda y oferta inversa (es decir P en funcion de Q). Encuentre: (i) el precio y cantidad de equilibrio, y: (ii) el excedente del consumidor, del productor y las ganancias totales para la sociedad. Gr�fique e interprete econ�micamente estos conceptos.
\item[(b)] Suponga ahora que el gobierno impone un impuesto por unidad consumida igual a 10 pesos; es decir por cada unidad que se consume, los
consumidores deben pagar el precio de mercado $P$ mas $\$ 10$ de impuestos. Encuentre: (i) la funci�n de demanda inversa, (ii) el precio y cantidad de �
equilibrio, (iii) el precio que efectivamente pagaran los consumidores, (iv) �
el excedente del consumidor, del productor, la recaudaci�n total del gobierno y las ganancias totales para la sociedad. Comente.
\end{itemize}
\end{document}