\documentclass[10pt,a4paper]{article}
\usepackage[latin1]{inputenc}
\usepackage[spanish]{babel}
\usepackage{amsmath}
\usepackage{amsfonts}
\usepackage{amssymb}
\usepackage{graphicx}
\usepackage[left=4cm,right=3cm,top=4cm,bottom=3cm]{geometry}
\author{Andr�s N. Dur�n}
\title{Econom�a Matem�tica}
\date{02 de Abril del 2018}
\begin{document}
\textbf{1. Topolog�a en IR} Suponga que usted tiene una funci�n de utilidad dada por $ u(x)=10x_{1}^{2}+x_{2}. $ Los precios que enfrenta son $ p_{1}=2 $ y $ p_{2}=1 $ para el bien 1 y 2 respectivamente. Su ingreso es igual a 100. Suponga adem�s que por temas de oferta en el mercado, no se puede demandar m�s de 40 unidades del bien 2, por tanto su conjunto presupuestario es:

\begin{equation}
\left( \dfrac{C_{1}}{C_{2}}\right) = \beta R
\end{equation}

\begin{equation}
C(p_{1},p_{2},m):=\left\lbrace x \in IR_{+}^{2}: 2x_{1} + x_{2} \leq 100, x_{2}\leq 40 \right\rbrace 
\end{equation}
\begin{itemize}
\item[(a)] Demuestre que $ u(x) $ es continua para todo $ a \in IR_{+}^{2} $
\item[(b)] Demuestre que $ C $ es acotado.
\end{itemize}


\textbf{2. Convergencia:} Demuestre que la secuencia $ X = \left\lbrace x_{n} \right\rbrace = \dfrac{2n}{n^{2}+1} $ converge a cero.

\textbf{3. Homogeneidad:} Considere la funci�n \textbf{CES} (Elasticidad de Sustituci�n Constante) para $ k $ variables:
\begin{equation}
f(x)=A\left[ \sum_{i=1}^{k} a_{i} x_{i}^{-\rho} \right]^{\dfrac{-\beta}{\rho}} 
\end{equation}

Para que valores de $ \beta $ esta funci�n es HG0, HG1 y HGN.\\

\textbf{4. Beneficios de la empresa:} La funci�n de beneficios de una empresa es $ B(x,y)=108 x^{\dfrac{1}{2}}y^{\dfrac{1}{3}} $, donde $ x,y $ son las cantidades invertidas respectivamente en la producci�n de dos art�culos $ A $ y $ B $. La producci�n actual es $ (x,y)=(4,27) $, pero la empresa dispone de $ 0,3 $u.m. para aumentar la producci�n � Le convendr� m�s destinarlas al art�culo A o al art�culo B?

\textbf{5. M�ximos o M�nimos} Considere la funci�n $ f: R^{2} \rightarrow R $ definida por $ f\left( x_{1},x_{2}\right) = x_{1}^{3}-x_{2}^{3}+9x_{1}x_{2}  $. Encuentre todos los puntos cr�ticos de $ f $ y determine, de ser posible, si constituyen m�ximos o m�nimos locales.
\end{document}

