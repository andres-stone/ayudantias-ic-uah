\documentclass[10pt,a4paper]{article}
\usepackage[latin1]{inputenc}
\usepackage[spanish]{babel}
\usepackage{amsmath}
\usepackage{amsfonts}
\usepackage{amssymb}
\usepackage{graphicx}
\usepackage[left=4cm,right=3cm,top=4cm,bottom=3cm]{geometry}
\author{Andr�s N. Dur�n}
\title{Econom�a Matem�tica}
\date{14 de Mayo del 2018}
\begin{document}
\maketitle
\textbf{1.} Sea $ a \in IR $ un par�metro. Considere la siguiente funci�n:
\begin{align*}
f(x;a)=2x^{1/2}-ax+a^{\beta}
\end{align*}
Use el teorema de la Envolvente para determinar el efecto de un cambio \textit{infinitisimal} de $ a$ en el m�ximo valor de la funci�n.\\


\textbf{2.} Pepe vive dos per�odos de tiempo; Su juventud, o el presente, y su vejez, o el futuro, (per�odo 2). La utilidad de Pepe depende de cu�nto consume en ambos per�odos:
\begin{align*}
U(c_{1},c_{2}):= u(c_{1})+\beta u(c_{2})
\end{align*}
Donde $u(c_{t})$ es la utilidad derivada del consumo $c_{t}$ en el per�odo $t = 1, 2$
y $\beta \in (0, 1)$ es el factor de descuento (valoraci�n del futuro). Pepe dispone
de $W$ unidades monetarias que puede usar tanto para consumir durante su juventud como para ahorrar en una AFP. Por cada unidad monetaria ahorrada durante su juventud, la AFP le devuelve a Pepe obtiene$ R > 1$
unidades monetarias durante su vejez. En resumen: 
\begin{align*}
c_{1}+s&=W\\
c_{2}&=Rs
\end{align*}
donde $s$ es la cantidad ahorrada en el per�odo 1. El objetivo de Pepe es maximizar utilidad de toda su vida. Asuma que $u'(c_{t}) > 0$.
\begin{itemize}
\item[(a)] Exprese la utilidad de Pepe en t�rminos de $s$, $W$ y $R$.
\item[(b)] Plantee el problema que debe resolver Pepe.
\item[(c)] Obtenga la condici�n de primer orden. Interprete la misma en t�rminos
econ�micos.
\item[(d)] Imponga una condici�n sobre $u(c_{t})$ para que la condici�n de primer orden sea suficiente.
\item[(e)] Demuestre que puede usar el teorema de la funci�n impl�cita y escribir
el ahorro optimo de Pepe $s^{*}$ como funci�n de los par�metros del problema.
\item[(f)] Use el resultado del apartado (e) para obtener las siguientes est�ticas comparativas:
\begin{align*}
\dfrac{\partial s^{*}}{\partial W}, \dfrac{\partial s^{*}}{\partial R}, \dfrac{\partial s^{*}}{\partial \beta}
\end{align*}
Explique los resultados obtenidos desde un punto de vista econ�mico.
\item[(g)] Obtenga la funci�n de m�xima utilidad(funci�n de utilidad indirecta) de Pepe:
\begin{align*}
V(W,R,\beta)
\end{align*}
\item[(h)] Use el teorema de la Envolvente para obtener:
\begin{align*}
\dfrac{\partial V(W,R,\beta)}{\partial W}, \dfrac{\partial V(W,R,\beta)}{\partial R}, \dfrac{\partial V(W,R,\beta)}{\partial \beta}
\end{align*}

\end{itemize}

\end{document}