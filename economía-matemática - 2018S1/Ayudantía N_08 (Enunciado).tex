\documentclass[10pt,a4paper]{article}
\usepackage[latin1]{inputenc}
\usepackage[spanish]{babel}
\usepackage{amsmath}
\usepackage{amsfonts}
\usepackage{amssymb}
\usepackage{graphicx}
\usepackage[left=4cm,right=3cm,top=4cm,bottom=3cm]{geometry}
\author{Andr�s N. Dur�n}
\title{Econom�a Matem�tica}
\date{11 de Junio del 2018}
\begin{document}
\maketitle
\textbf{1.} Resuelva, utilizando el m�todo de Lagrange, el siguiente problema de maximizaci�n:
\begin{align*}
\max_{x} U(x) &= \max_{x}\left\lbrace 2x_{1}^{0.5}+x_{2}^{0.5}+x_{3} \right\rbrace \\
x \in F&:= \left\lbrace x \in R^{3}: \dfrac{1}{5}x_{1}+\dfrac{1}{4}x_{2}+x_{3}=40; x_{2}=100\right\rbrace 
\end{align*}

\textbf{2.} Un estudiante desea asignar 60 horas semanales para estudiar dos asignaturas diferentes: econom�a matem�tica (asignatura 1) y comercio internacional (asignatura 2). La calificaci�n $ g_{i} $ en la asignatura $ i=1,2 $ depende del tiempo dedicado por el individuo a tal asignatura
\begin{align*}
g_{1}=20+20\sqrt{x_{1}} & & g_{2}=-80+3x_{2}
\end{align*}

Donde $ x_{i} $ es el tiempo de estudio dedicado a la asignatura $ i=1,2 $. El objetivo del estudiante es maximizar el promedio de sus calificaciones. Finalmente, el estudiante debe agotar todo su tiempo en el estudio de estas dos asignaturas.
\begin{itemize}
\item[i.] Escriba formalmente el problema que debe resolver el estudiante.
\item[ii.] Escriba la funci�n de Lagrange.
\item[iii.] Obtenga las condiciones de primer orden y encuentre la soluci�n al problema.
\end{itemize}
\textbf{3.} Considere una econom�a que dispone de 100 unidades de trabajo y que puede producir los bienes $ x_{1} $ y $ x_{2} $. Por cada unidad de $ x_{i}^{2} $ producidas se requieren $ x_{i}^{2} $ unidades de trabajo para $ i=1,2 $. Adem�s, sabe que la sociedad desea maximizar la siguiente funci�n objetivo
\begin{align*}
F(x_{1},x_{2}) = ax_{1}+bx_{2}
\end{align*}

Sujeta a la restricci�n de recursos, Asuma que $ a $ y $ b  $ son constantes estrictamente positivas. Utilice el m�todo de los multiplicadores de Lagrange para encontrar las cantidades �ptimas a producir de los bienes$ x_{1} $ y $ x_{2} $. Interprete sus resultados.
\end{document}