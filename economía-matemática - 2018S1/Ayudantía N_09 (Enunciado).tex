\documentclass[10pt,a4paper]{article}
\usepackage[latin1]{inputenc}
\usepackage[spanish]{babel}
\usepackage{amsmath}
\usepackage{amsfonts}
\usepackage{amssymb}
\usepackage{graphicx}
\usepackage[left=4cm,right=3cm,top=4cm,bottom=3cm]{geometry}
\author{Andr�s N. Dur�n}
\title{Econom�a Matem�tica}
\date{18 de Junio del 2018}
\begin{document}
\maketitle
\textbf{1.} Resuelva, utilizando el m�todo de Lagrange, el siguiente problema de maximizaci�n:
\begin{align*}
\max_{x,c}  &= \left\lbrace cx \right\rbrace \\
s.&a\\
c+wx&=wT+b\\
T-x\leq&h
\end{align*}

Donde $ c $ es el consumo del bien agregado, $ x $ las horas disponible de ocio, $ b > 0$ es herencia ex�gena, $ T $ el tiempo dedicado para el ocio y trabajo y $ w $ el salario por hora. El individuo desea maximizar el consumo y el ocio, para esto construya el lagrangeano y ocupe las condiciones de Kuhn - Tucker.\\

\textbf{2.} Resuelva el siguiente problema:
\begin{align*}
\max_{x} U(x):=& \max_{x}\left\lbrace ax_{1}+ln x_{2} \right\rbrace \\
x \in X(p_{1},p_{2},m):=& \left\lbrace x \in IR^{2}: m\geq p*x, x\geq 0 \right\rbrace 
\end{align*}
Donde $ p=(p_{1},p_{2}) $ , $ x=(x_{1},x_{2}) $ y a es una constante estrictamente positiva. Recuerde incluir las restricciones de no negatividad. �C�mo se relaciona el consumo del bien 1 con el ingreso m?
\end{document}