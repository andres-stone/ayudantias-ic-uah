\documentclass[10pt,a4paper]{article}
\usepackage[latin1]{inputenc}
\usepackage[spanish]{babel}
\usepackage{amsmath}
\usepackage{amsfonts}
\usepackage{amssymb}
\usepackage{graphicx}
\usepackage[left=4cm,right=3cm,top=4cm,bottom=3cm]{geometry}
\author{Andr�s N. Dur�n}
\title{Econom�a Matem�tica}
\date{16 de Abril del 2018}
\begin{document}
\maketitle
\textbf{1.} Sea la tecnolog�a definida por una funci�n Cobb-Douglas:
\begin{align*}
Q=F(K,L)&=20K^{1/3}L^{2/3}
\end{align*}
Donde $ K $ es el stock de capital y $ L $ las unidades de mano de obra. Su utilizando $ K=1.000.000 $ y $ L=64 $ se produce $ Q=32.000 $. Si el stock de capital aumenta en 100 unidades adicionales, mientras que la mano de obra se mantiene en 64. �Cu�l es el incremento real de producci�n?. �Cu�l es el incremento aproximado de la producci�n?\\


\textbf{2.} Si $ P(t) $ es el producto interno bruto de un pa�s en un tiempo t, �Qu� es la derivada $ \dfrac{\partial P}{\partial t} $?\\

\textbf{3.} El IPC de un cierto pa�s en un instante t (expresado en a�os) viene dado por la formula
\begin{align*}
P&=e^{\sqrt{(1+t/50)^{3}}}
\end{align*}
\begin{itemize}
\item[(a)] Calcula la inflaci�n del pa�s, es decir, el porcentaje de aumento de los precios:
\begin{align*}
I&=\dfrac{100}{P}\dfrac{\partial P}{\partial t}
\end{align*}
�Qu� tanto por ciento de inflaci�n se tiene en $ t=0 $?
\item[(b)] Estudia el comportamiento de la inflaci�n en $ t=0 $. �Est� aumentando o disminuyendo?
\item[(c)] Calcula la tasa de incremento de la inflaci�n en el pa�s, es decir,
\begin{align*}
T&=\dfrac{100}{I} \dfrac{\partial I}{\partial t}
\end{align*}
�Cu�nto vale en $ t=0 $?
\item[(d)] �C�mo var�a la tasa de incremento de la inflaci�n del pa�s?, �crece o decrece?
\end{itemize}

\textbf{4.} Sea la siguiente funci�n cuadr�tica definida en $ f: IR \rightarrow IR $
\begin{align*}
f(x)&=ax^{2}+bx+c
\end{align*}

Demuestre utilizando la definici�n de \textbf{convexidad/c�ncavidad} en $ IR $ que la convexidad depende exclusivamente de $ a \in IR$.
\end{document}