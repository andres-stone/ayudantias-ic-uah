\documentclass[10pt,a4paper]{article}
\usepackage[latin1]{inputenc}
\usepackage[spanish]{babel}
\usepackage{amsmath}
\usepackage{amsfonts}
\usepackage{amssymb}
\usepackage{graphicx}
\usepackage[left=4cm,right=3cm,top=4cm,bottom=3cm]{geometry}
\author{Andr�s N. Dur�n}
\title{Econom�a Matem�tica}
\date{22 de Marzo de 2018}
\begin{document}
\maketitle
\textbf{1. L�gica Proposicional:} Suponga la siguiente proposici�n:
\begin{equation}
p\Rightarrow \left[ p \wedge \sim \left( q \vee r \right)  \right] 
\end{equation}

Utilizando propiedades de absorci�n demuestre que la proposici�n anterior es equivalente a $\sim p \vee \left( \sim q \wedge \sim r \right) $.\\

\textbf{2. Conjuntos:} Suponga dos conjuntos cualesquiera $A$, $B$. Utilizando propiedades de conjuntos, demuestre la siguiente equivalencia.
\begin{equation}
A\bigtriangleup B =\left( A \ B \right) \cup \left( B \ A \right) \equiv \left( A \cup B \right) \ \left( B \cap A \right)
\end{equation}

\textbf{3. Cardinalidad de Conjuntos:} Suponga que una encuesta a 92 estudiantes de econom�a sobre sus preferencias respecto a microeconom�a, macroeconom�a
y econometr�a, arrojo que 53 de estos les gusta la microeconom�a, 46 la macroeconom�a
y 35 la econometr�a, sin embargo sus respuestas fueron indecisas, pues 22 dijeron que
les gustaba tanto la microeconom�a como la macroeconom�a, 16 la micro y la econometr�a, y 14 la
macroeconom�a y la econometr�a. Ahora bien, dado esto, �Es posible computar a cuantos estudiantes les gusta las tres �reas? y �Cu�ntos prefiere una de ellas respecto a las otras?\\

\textbf{4. Precios de competencia:} Sea $ \bar{p}=\left( p_{1},p_{2},p_{3} \right)  $ el vector de precio en el que se venden tres productos $A$, $B$ y $C$. Se esima que la oferta y la demanda de cada uno de ellos viene dada por:
\begin{align*}
S_{A}&=15p_{1}+p_{2}+p_{3}-13 & D_{A}&=70-8p_{1}-p_{2}-p_{3}\\
S_{B}&=p_{1}+20p_{2}+10p_{3}-10 & D_{B}&= 93-2p_{1}-4p_{2}-p_{3}\\
S_{C}&=10p_{1}+15p_{2}+30p_{3}-50 & D_{C}&= 107-p_{1}-3p_{2}-5p_{3}\\
\end{align*}

Donde $ p_{1} $, $ p_{2} $ y $ p_{3} $ son los precios unitarios de los productos $ A $, $ B $ y $ C $, respectivamente. Calcula el vector $ \bar{p} $ para este sistema de ecuaciones.\\

\textbf{5. Topolog�a en IR:} Suponga que usted tiene un funci�n de utilidad dada por $u(x)=10x_{1}^{2}+x_{2}$. Los precios que enfrenta son $p_{1}=2$ y $p_{2}=1$ para el bien 1 y 2 respectivamente. Su ingreso es igual a 100. Suponga adem�s que por temas de oferta en el mercado, no se puede demandar m�s de 40 unidades del bien 2, por tanto su conjunto presupuestario es:
\begin{equation}
C(p_{1},p_{2},m):= \left\lbrace x \in IR_{+}^{2}: 2x_{1}+x_{2} \leq 100, x_{2}\leq 2 \right\rbrace 
\end{equation}
\begin{itemize}
\item[(a)] Demuestre que $u(x)$ es continua para todo $ a \in IR_{+}^{2}$.
\item[(b)] Demuestre que $C$ es acotado.
\end{itemize}
\end{document}