\documentclass[10pt,a4paper]{article}
\usepackage[latin1]{inputenc}
\usepackage[spanish]{babel}
\usepackage{amsmath}
\usepackage{amsfonts}
\usepackage{amssymb}
\usepackage{graphicx}
\usepackage[left=4cm,right=3cm,top=4cm,bottom=3cm]{geometry}
\author{Andr�s N. Dur�n}
\title{Econom�a Matem�tica}
\date{23 de Abril del 2018}
\begin{document}
\maketitle
\textbf{1.} La demanda del bien 1 es $Q = f(p_{1}, p_{2}, I)$, donde $p_{i}$ es el precio del
bien $i = 1, 2$ e $I$ es el ingreso total de la poblaci�n que consume el bien 1. El estado tecnol�gico de la econom�a afecta tanto al ingreso como al precio del bien 2. Espec�ficamente $p_{2} = F(a)$ y $I = G(a)$ donde $a \in [0,\infty )$ representa
el estado de la tecnolog�a. Suponemos que:
\begin{align*}
\dfrac{dp_{2}}{da}<0, \dfrac{dI}{da}>0
\end{align*}

Es decir, suponemos que el precio del bien 2 disminuye y el ingreso aumenta cuando el estado tecnol�gico mejora.
\begin{itemize}
\item[(a)] Asuma que el bien 1 es un bien normal. Determine el signo de $\partial Q/\partial I$.
\item[(b)]Asumiendo que el bien 1 es un bien normal determine el signo de $\partial Q/\partial p_{1}$.
\item[(c)] Suponga que el bien 1 y el 2 son sustitutos. Determine el signo de $\partial Q/\partial p_{2}$.
\item[(d)] Suponga que el bien 1 y el 2 son complementarios. Determine el signo
de $\partial Q/\partial p_{2}$.
\item[(e)] Asumiendo que el bien 1 y el 2 son sustitutos, �Aumenta o disminuye la demanda del bien 1 cuando el estado tecnol�gico mejora?
\item[(f)] Asumiendo que el bien 1 y el 2 son complementarios, �Aumenta o disminuye la demanda del bien 1 cuando el estado tecnol�gico mejora?
\end{itemize}


\textbf{2.} La demanda y oferta del bien x, est�n dadas por:
\begin{align*}
D&=a-b(P+\tau)\\
S&=\alpha +\beta P
\end{align*}
Siendo $ \tau $ un impuesto por unidad consumida. Adem�s $ a,b,\alpha $ y $ \beta $ son constante positvas.
\begin{itemize}
\item[(a)] Escriba la condici�n de equilibrio del mercado.
\item[(b)] Muestre que la condici�n de equilibrio define impl�citamente a $ P $ como funci�n de $ \tau $
\item[(c)] Obtenga usando el teorema de la funci�n impl�cita $ dp/d\tau $. Corrobore su resultado resolviendo la condici�n de equilibrio de mercado para $ P $ y luego encontrando $ dp/d\tau $ expl�citamente.
\item[(d)] Suponga que la demanda y oferta se expresan en forma general de la siguiente manera.
\begin{align*}
D&=F(P+\tau)\\
S&=G(P)
\end{align*}
Donde $ F $ y $ G $ son funciones diferenciables con $ F'<0 $ y $ G'>0 $. Repita los incisos anteriores.
\end{itemize}

\textbf{3.} Suponga que una empresa perfectamente competitiva vende su producto, cuya cantidad se denota por $ y $, a un precio $ p > 0 $ y compra trabajo, denotado por $ x\geq  0$, a un precio $ w >0 $. La funci�n de producci�n de la empresa es $ y= F(x,a) $ donde $ a \in IR $ es un parametro que mide el estado tecnol�gico de la misma. Asuma que:
\begin{align*}
\dfrac{\partial y}{\partial a}=F_{a}(x,a)>0, \dfrac{\partial y}{\partial x}=F_{x}(x,a)>0, \dfrac{\partial y}{\partial a \partial x}=F_{ax}(x,a)>0
\end{align*}
Y que $ F $ es dos veces diferenciable respecto a x.
\begin{itemize}
\item[(a)] Escriba la funci�n de beneficios de la empresa, $ \pi (x;p,w,a) $.
\item[(b)] Imponga una condici�n que garantice que $ \pi $ sea estrictamente c�ncava en $ x $.
\item[(c)] Obtenga la condici�n de primer orden para maximizar beneficios. �Es esta condici�n suficiente en este caso?
\item[(d)] Demuestre que puede usar el teorema de la funci�n impl�cita y escribir la cantidad �ptima de trabajo $ x^{*} $ como funci�n de los par�metros del problema. Esto es:
\begin{align*}
x^{*}=x(p,w,a)
\end{align*}
\item[(e)] Use el resultado de $ (d) $  y las condiciones de primer orden para obtener las siguientes est�ticas comparativas:
\begin{align*}
\dfrac{\partial x}{\partial p}, \dfrac{\partial x}{\partial w},\dfrac{\partial x}{\partial a}
\end{align*}
Explique los resultados obtenidos desde un punto de vista econ�mico.
\end{itemize}
\end{document}